\section{Theorie}
\label{sec:Theorie}

% TODO: überarbeiten
In diesem Versuch wird die Funktionsweise eines Reinst-Germanium-Detektors diskutiert
und verschiedene Proben anhand aufgenommener Spektren charakterisiert.
Dazu werden im Folgenden die verschiedenen Mechanismen zur Energiedeposition von
$\gamma$-Strahlung in Materie erläutert, der Aufbau eines Halbleiter-Detektors
dargestellt und ein typisches Spektrum einer $\gamma$-Quelle diskutiert.

\subsection{Photo-Effekt}
\label{sec:Photo-Effekt}

\subsection{Compton-Effekt}
\label{sec:Compton-Effekt}

\subsection{Paarerzeugung}
\label{sec:Paarerzeugung}

\subsection{Funktionsweise eines Halbleiter-Detektors}
\label{sec:HLDetektor}

\subsection{Spektrum, Aktivität und Energie einer Gamma-Quelle}
\label{sec:TypischeQuelle}

% \begin{figure}
%   \centering
%   \includegraphics[height=5.5cm]{content/Bild.png}
%   \caption{Bilduterschrift}
%   \label{fig:Bild}
% \end{figure}
