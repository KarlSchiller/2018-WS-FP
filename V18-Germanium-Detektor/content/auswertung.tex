\newpage
\section{Auswertung}
\label{sec:Auswertung}

%-------------------------------------------------------------------------------
\subsection{Energiekallibrierung}
\label{sec:Energiekallibrierung}

Für eine Energiekalibrierung wird das Spektrum der \ce{^{152}Eu}-Quelle
aufgezeichnet. Dies ist in Abbildung \ref{plt:Eu-Spektrum} dargestellt. Die zu
erkennenen Peaks werden mit dem \textit{find.peaks}-Paket von python
herausgefiltert und dem erwartetem Europium-Spektrum zugeordnet. Dies ist in
Tabelle \ref{tab:anleitung_eu} dargestellt.
\begin{figure}[htb]
    \centering
    \includegraphics[width=0.8\textwidth]{build/orginal_Eu_log.pdf}
  \caption{Vom Germanium Detektor aufgenommenes Spektrum der $^{152}$Europium Quelle mit
  logarithmischer y-Achse, aufgenommen innerhalb einer Stunde.}
  \label{plt:Eu-Spektrum}
\end{figure}
Anhand der Energien und der Zählrate wurde eine lineare Ausgleichsrechnung
mit Hilfe der Formel
\begin{align*}
  E = m \cdot x + b
\end{align*}
durchgeführt. Dazu wurde das Paket \textit{curve\_fit} verwendet. Die
Ausgleichsrechnung ist in Abbildung \ref{plt:Eu-Spektrum} dargestellt.
\begin{figure}[htb]
  \centering
  \includegraphics[width=0.8\textwidth]{build/kalibation.pdf}
  \caption{Zugeordnete Energien aufgetragen gegen die Bin-Indizes zur
  Kalibrierung des Detektors für die beobachteten Peaks (blau). Zudem ist eine
  lineare Regression der Werte (rot) aufgetragen.}
  \label{plt:eichung}
\end{figure}
Dabei entspricht $x$ der Bin-Indizes. Für die Parameter $m$ und
$b$ ergeben sich:
\begin{align*}
	m &= \num{0.4054(35)} \\
  b &= \SI{-11.22(731)}{\kilo\electronvolt}
\end{align*}
\begin{table}
	\centering
	\caption{Gegebene Werte zur Kalibrierung des Germanium-Detektors \cite{anleitung}.}
	\label{tab:anleitung_eu}
	\begin{tabular}{
		S[table-format=1.0]
		S[table-format=2.1]
		S[table-format=1.0]
		}
	\toprule
		{Energie $E\;/\;\si{\si{\kilo\electronvolt}$}} &
		{Bin-Index $i$} &
		{Emis.-Wahr. W} \\
	\midrule
		 122 &  28.6 &  378 \\
		 245 &  7.6 &  691 \\
		 296 &  0.4 &  742 \\
		 344 &  26.5 &  861 \\
		 411 &  2.2 &  1108 \\
		 444 &  3.1 &  1220 \\
		 678 &  2.0 &  1716 \\
		 689 &  0.9 &  1790 \\
		 779 &  12.9 &  1939 \\
		 867 &  4.2 &  2185 \\
		 964 &  14.6 &  2400 \\
		 1005 &  0.6 &  2502 \\
		 1086 &  10.2 &  2703 \\
		 1112 &  13.6 &  2765 \\
		 1299 &  1.6 &  3014 \\
		 1408 &  21.0 &  3501 \\
	\bottomrule
	\end{tabular}
\end{table}
\FloatBarrier

%-------------------------------------------------------------------------------
\subsection{Vollenergienachweiseffizienzmessung des Detektors}
\label{sec:Effizienzmessung}

Zur Berechnung der Detektoreffizienz wird Formel
\eqref{eqn:Vollenergie-Nachweiseffizienz} verwendet. Dafür wurde zuerst der
Raumwinkel anhand Formel \eqref{eqn:Raumwinkel} berechnet. Mit
$a = \SI{37.5(10)}{\milli\meter}$ und $r = \SI{22.5(10)}{\milli\meter}$ ergibt sich
\begin{align*}
  \frac{\Omega}{4\pi} = \num{0.01558(34)}.
\end{align*}
Dabei wurde $r$ so gewählt, dass der wahrscheinlichste Wechselwirkungspunkt
\SI{1.5}{\centi\meter} innerhalb des Germaniums liegt und der Abstandshalter
zwischen Probe und Detektor \SI{7.31}{\centi\meter} lang ist.
Die Aktivität wurde aus dem Wissen errechnet, dass am 01.10.2000 die Aktivität
der Europium-Probe bei $A = \SI{4130(60)}{\becquerel}$ beträgt. Mithilfe der
Halbwertszeit $t_{\sfrac{1}{2}} = \SI{4943(5)}{\day}$ folgt für den Messtag
\begin{align*}
	A_\text{Messtag} = A_0 \exp\left(-\frac{\ln{2} \cdot\Delta t}{t_{\sfrac{1}{2}}}\right)=\SI{1633(24)}{\becquerel}
	.
\end{align*}
Der Peakinhalt wurde mit einem Fit durch eine Gaußfunktion der Form
\begin{align*}
	f\left(x\right) = h\cdot \exp{\frac{(x-\mu)^2}{2\sigma^2}} + a
\end{align*}
für jede Energie durchgeführt. Hierbei beschreibt $h$ die Höhe des Peaks, $\mu$
den (um Bin-Indizes der Peaks zu korrigieren) Mittelwert, $\sigma$ die
Standardabweichnug und $a$ einen Parameter zur Berücksichtigung des Untergrunds.
bezeichnet. Die Parameter der Fits für jedes Bin sind in Tabelle
\ref{tab:gauss_parameter} nachzulesen.
\begin{table}
	\centering
	\caption{Parameter des durchgeführten Gauss-Fits pro Bin. Dabei ist $\mu$ der Mittelwert, $\sigma$ die Standardabweichnug, $h$ die Höhe und a der Energieoffset.}
	\label{tab:gauss_parameter}
	\begin{tabular}{
		S[table-format=1.2] @{${}\pm{}$} S[table-format=1.2]
		S[table-format=1.2] @{${}\pm{}$} S[table-format=1.2]
		S[table-format=1.2] @{${}\pm{}$} S[table-format=1.2]
		S[table-format=1.2] @{${}\pm{}$} S[table-format=1.2]
		}
	\toprule
		\multicolumn{2}{c}{$a$} &
		\multicolumn{2}{c}{$h_i$} &
		\multicolumn{2}{c}{$\mu_i$} &
		\multicolumn{2}{c}{$\sigma_i$} \\
	\midrule
		 68.14 &  1.81 &  18.53 &  2.59 &  384.09 &  1.61 &  11.59 &  2.24 \\
		 28.53 &  0.62 &  36.42 &  209088.64 &  691.33 &  608.71 &  0.26 &  477.23 \\
		 25.80 &  0.56 &  30.13 &  3.79 &  741.15 &  0.20 &  1.39 &  0.20 \\
		 21.02 &  1.00 &  1367.21 &  6.25 &  860.92 &  0.01 &  1.58 &  0.01 \\
		 16.76 &  0.57 &  118.32 &  3.36 &  1108.16 &  0.06 &  1.75 &  0.06 \\
		 16.54 &  0.50 &  7.93 &  3.03 &  1219.87 &  0.74 &  1.69 &  0.76 \\
		 15.21 &  0.42 &  25.36 &  3.66 &  1715.98 &  0.14 &  0.81 &  0.13 \\
		 14.05 &  0.41 &  13.86 &  3.53 &  1789.76 &  0.26 &  0.89 &  0.26 \\
		 12.66 &  0.53 &  180.12 &  2.63 &  1939.14 &  0.04 &  2.39 &  0.04 \\
		 27.24 &  6.78 & -19.01 &  6.23 &  2205.59 &  5.66 &  26.95 &  12.42 \\
		 6.42 &  0.66 &  148.74 &  2.84 &  2398.69 &  0.07 &  3.00 &  0.07 \\
		 5.18 &  0.29 &  7.98 &  1.39 &  2500.53 &  0.50 &  2.48 &  0.51 \\
		 6.44 &  0.62 &  70.59 &  2.39 &  2701.61 &  0.14 &  3.64 &  0.15 \\
		 5.67 &  0.55 &  104.52 &  2.30 &  2766.12 &  0.08 &  3.22 &  0.08 \\
		 3.66 &  0.29 &  9.67 &  1.40 &  3015.25 &  0.42 &  2.57 &  0.44 \\
		 0.69 &  0.45 &  107.38 &  1.70 &  3500.68 &  0.07 &  3.81 &  0.07 \\
	\bottomrule
	\end{tabular}
\end{table}
\FloatBarrier

Aus diesen ergibt sich der Peakinhalt $Z_i$ des Peaks $i$ durch Integration
über eine Gaußkurve:
\begin{align*}
  Z_i = \sqrt{2\pi} h_i \sigma_i
\end{align*}
An diese in Tabelle \ref{tab:det_eff} dargestellten Werte wird ein Fit der Form
\begin{align*}
  Q(E) = a \cdot (E - b)^e + c
\end{align*}
durchgeführt. Dabei wurden nur solche Energien betrachtet, die über
\SI{150}{\kilo\electronvolt} liegen.
Der Fit ist in Abbildung \ref{plt:eff} dargestellt und ergab die Parameter:
\begin{align*}
  a &= -\SI{0.01(7)}{\per\kilo\electronvolt\per\becquerel} \\
  b &= \SI{114.0(6)}{\kilo\electronvolt} \\
  c &= \num{0.5(4)} \\
  e &= \num{0.5(7)}
\end{align*}
\begin{figure}[htb]
  \centering
  \includegraphics[width=0.8\textwidth]{build/efficiency.pdf}
  \caption{Fit zur Effizienzbestimmung des Detektors anhand der zuvor
  berechneten Werten der Efizienz anhand der Energien.}
  \label{plt:eff}
\end{figure}

\begin{table}
	\centering
	\caption{Peakhöhe, Energie und Detektoreffizenz als Ergebnis des Gaußfits.}
	\label{tab:det_eff}
	\begin{tabular}{
		S[table-format=1.2] @{${}\pm{}$} S[table-format=1.2]
		S[table-format=1.2]
		S[table-format=1.2] @{${}\pm{}$} S[table-format=1.2]
		}
	\toprule
		\multicolumn{2}{c}{$Z_i$} &
		{E_i} &
		\multicolumn{2}{c}{$Q \ \si{becquerel}$} \\
	\midrule
		 538.41 &  128.48 &  120.24+/-0.68 &  0.74 &  0.18 \\
		 24.06 &  144843.44 &  249.41+/-255.92 &  0.12 &  749.00 \\
		 104.94 &  20.29 &  270.36+/-0.08 &  10.31 &  2.00 \\
		 5423.10 &  38.12 &  320.71+/-0.00 &  8.04 &  0.13 \\
		 519.47 &  22.69 &  424.66+/-0.02 &  9.28 &  0.43 \\
		 33.65 &  19.80 &  471.63+/-0.31 &  0.43 &  0.25 \\
		 51.61 &  11.13 &  680.21+/-0.06 &  1.01 &  0.22 \\
		 30.91 &  12.06 &  711.23+/-0.11 &  1.35 &  0.53 \\
		 1079.54 &  24.38 &  774.04+/-0.02 &  3.29 &  0.09 \\
		-1284.03 &  725.94 &  886.06+/-2.38 & -12.02 &  6.80 \\
		 1118.21 &  33.15 &  967.25+/-0.03 &  3.01 &  0.10 \\
		 49.64 &  13.42 &  1010.07+/-0.21 &  3.25 &  0.88 \\
		 644.04 &  33.97 &  1094.61+/-0.06 &  2.48 &  0.14 \\
		 844.48 &  28.90 &  1121.73+/-0.03 &  2.44 &  0.09 \\
		 62.29 &  13.92 &  1226.48+/-0.18 &  1.53 &  0.34 \\
		 1024.99 &  25.27 &  1430.57+/-0.03 &  1.92 &  0.05 \\
	\bottomrule
	\end{tabular}
\end{table}
\FloatBarrier

%-------------------------------------------------------------------------------
\subsection{Bestimmung der Detektoreigenschaften}
\label{sec:Detektoreigenschaften}
Für diesen Teil der Auswertung wurde das Spektrum eines Cäsium-Strahlers, in
Abbildung \ref{plt:Cs} zu sehen, aufgenommen.
\begin{figure}
    \centering
    \includegraphics[width=0.8\textwidth]{build/spektrum_Cs_log.pdf}
  \caption{Vom Detektor aufgenommenes Spektrum der Cäsium-Quelle.}
  \label{plt:Cs}
\end{figure}
Hier wurden die charakteristischen
Peaks des Spektrums (Rückstreu- und Vollenergiepeak, sowie die Comptonkante)
identifiziert. Die Zuordnung dieser sind in Tabelle \ref{tab:Cs_char} nachzulesen. Der
Theoriwert des Vollenergiepeaks liegt bei \SI{661.59}{\kilo\electronvolt}
\cite{theorie}.
\begin{table}[htb]
	\centering
  \caption{Experimentell bestimmte charakteristische Peaks des Cs-Strahlers ahnand der Bin-Indizes und der Energie $E$.}
  \label{tab:Cs_char}
  \begin{tabular}{c
    S[table-format=4.0]
    S[table-format=3.2]}
    \toprule
    {} & {Index $i$} & {$E_i$ / keV} \\
    \midrule
    Rückstreupeak & 511 & 173,59 \\
    Compton-Kante & 1174 & 452,63 \\
    Vollenergiepeak & 1648 & 651,63 \\
    \bottomrule
  \end{tabular}
\end{table}
Eine Berechnung der Theoriewerte für Rückstreupeak und Comptonkante durch die
Energie des Vollenergiepeaks $E_{\text{Voll}}$ normiert auf $m_0\:c^2$ ergibt:
\begin{align*}
  E_\text{Compton, Theo} &= \frac{2\epsilon}{1+2\epsilon}\cdot E_\text{Voll} = \SI{477.27}{\kilo\electronvolt} \\
  E_\text{Rück, Theo} &= \frac{1}{1+2\epsilon}\cdot E_\text{Voll} = \SI{184.32}{\kilo\electronvolt}
\end{align*}
Diese Werte weichen wie folgt von den gemessenen Werten ab:
\begin{align*}
  \Delta E_\text{Compton} &= \SI{2.63}{\percent} \\
  \Delta E_\text{Rück} &= \SI{6.31}{\percent}
\end{align*}
Daraus folgt, dass der Rückstreupeak durch andere Effekte wie dem Photoeffekt
beeinflusst. Daraus ergibt sich eine Erhöhung der Energie.

Die Halbwerts- und Zehntelbreite des Vollenergiepeaks werden per Auge durch die Abbildung \ref{plt:halb}
abgeschätzt auf
\begin{align*}
  x_{\sfrac{1}{2}} &= \SI{2.2(2)}{\kilo\electronvolt} \\
  x_{\sfrac{1}{10}} &= \SI{4.0(3)}{\kilo\electronvolt}.
\end{align*}
\begin{figure}[htb]
  \centering
  \includegraphics[width=0.8\textwidth]{build/test_2.pdf}
  \caption{Bestimmung der Halbwerts- und Zehntelbreite des Vollenergiepeaks der Cäsiumquelle.}
  \label{plt:halb}
\end{figure}
Diese ergeben einen Quotienten von
\begin{align*}
  \frac{x_{\sfrac{1}{10}}}{x_{\sfrac{1}{2}}} = \num{1.818(387)}
\end{align*}


Im letzten Schritt dieses Auswertungsteils wird der Inhalt des Compton-Kontinuums
und des Vollenergiepeaks mit der Absorptionswahrscheilichkeit des Compton- und
Photoeffektes verglichen. Dafür wird die Formel
\begin{align*}
	p = 1 - \exp\left(-\mu l\right)
\end{align*}
mit der Länge des Detektors $l = \SI{3.9}{\centi\meter}$ und dem
Absorptionskoeffizienten $\mu$ verwendet. Dabei ergeben sich die
Absorptionskoeffizienten aus Abbildung \ref{plt:Cs_abs} zu:
\begin{align*}
	\mu_\text{Compton} &= \SI{0.38(3)}{\per\centi\meter} & p_\text{Compton} &= \SI{74(7)}{\percent} \\
	\mu_\text{Photo} &= \SI{0.002(3)}{\per\centi\meter} & p_\text{Photo} &= \SI{2.7(1)}{\percent}
\end{align*}
\begin{figure}
    \includegraphics[width=0.8\textwidth]{build/Cs_log.pdf}
  \caption{Vollenergiepeak und Compton-Kontinuum des Cs-Spektrums.}
  \label{plt:Cs_abs}
\end{figure}
Hier wird sowohl für das Compton-Kontinuum, als auch für den Vollenergiepeak ein
Fit durchgeführt. Als Fit-Funktion wird bei dem Compton-Kontinuum eine
Gaußfunktion angesetzt, während beim Vollenergiepeak eine Funktion der Form von
Formel \eqref{eqn:Compton-Wirkungsquerschnitt} verwendet wurde. Bei Durchführung
es Fits ergibt sich damei ein Parameter $\epsilon = \num{111}$. Die Peakhöhe des
untersuchten Compton-Kontinuums und des Vollenergiepeaks wird durch
Integration dieser Formel auf
\begin{align*}
  Z_\text{Compton} &= \num{9.0(8)e4} \\
  Z_\text{Voll} &= \num{3.39(7)e4}
\end{align*}
bestimmt.


%-------------------------------------------------------------------------------
\subsection{Aktivitätsbestimmung anhand eines Spektrums}
\label{sec:Aktivitätsbestimmung}
Im Folgenden wird anhand eines Spektrums zwischen Barium(\ce{^{133}Ba}) und Antimon(\ce{^{125}Sb})
unterschieden. Hierfür wurde das in Abbildung \ref{plt:bar} zu sehende
Spektrum verwendet, um wie in den Auswertungsteilen zuvor die Peaks den jeweiligen
Energien zugeordnet. Diese Zuordnungen sind in Tabelle \ref{tab:Ba_erwartet} zu
finden.
\begin{figure}[htb]
  \centering
  \includegraphics[width=0.8\textwidth]{build/Ba_Sb_orginal_log}
  \caption{Aufgenommenes Spektrum zur Bestimmung der Aktivität.}
  \label{plt:bar}
\end{figure}
\input{build/tables/Ba_erwartet.tex}
\FloatBarrier
Daraus ist dem Spektrum die \ce{^{152}Ba}-Quelle zuzuordnen, da die Energien der
Peaks gut zu dem von Barium zu erwartenden Spektrum passen. Einige Peaks, welche
charakteristisch für Barium sind, sind allerdings nicht gefunden worden.
Weitere Peaks, die klar zu erkennen sind und nicht zugeordnet wurden, sind auf
Untergrundstrahlung oder Verschmnutzungen der Probe zurückzuführen. Nach
Zuordnung der Peaks wurde ein Gauß-Fit verwendet, um die Peakhöhen der einzelnen
Bins zu erhalten. Nun wurde die Formel \eqref{eqn:Vollenergie-Nachweiseffizienz}
verwendet, um die Aktivität der Probe am Messtag zu bestimmen. Diese Berechnung
wird unter der Premisse durchgeführt, das alle Werte für die Energie überhalb
von $E_\text{i} = $\SI{150}{\kilo\electronvolt} liegt. Aus den Parametern des
Fits aus Tabelle \ref{tab:Ba} und den daraus Berechneten Werten in Tabelle
\ref{plt:aktivitaet_ba} wurde die Aktivität durch Bildung eines Mittelwertes
auf
\begin{align*}
  A = \SI{1417(32)}{\becquerel}
\end{align*}
bestimmt.
\input{build/tables/Ba.tex}
\begin{table}
	\centering
	\caption{Berechnete Aktivitäten für jeden Bin mit dazu benötigten Werten.}
	\label{plt:aktivitaet_ba}
	\begin{tabular}{
		S[table-format=2.2] @{${}\pm{}$} S[table-format=2.2]
		S[table-format=4.2] @{${}\pm{}$} S[table-format=3.1]
		S[table-format=5.2] @{${}\pm{}$} S[table-format=3.2]
		}
	\toprule
		\multicolumn{2}{c}{$Z_i$} &
		\multicolumn{2}{c}{$E_i$\;/\;\si{\kilo\electronvolt }} &
		\multicolumn{2}{c}{$A_i$\;/\;\si{\becquerel }} \\
	\midrule
		 62.81 &  4.76 &  7523.02 &  154.7 & 22869 &  610.68 \\
		 41.19 &  1.03 &  30.73 &  33.0 & 0 & 0 \\
		 14.07 &  0.49 &  1123.98 &  17.5 & 1625 &  36.98 \\
		 10.72 &  0.51 &  2645.61 &  18.3 & 1462 &  32.28 \\
		 6.39 &  0.82 &  7351.38 &  30.9 & 1371 &  32.41 \\
		 2.56 &  0.23 &  950.79 &  8.9 & 0 & 0 \\
	\bottomrule
	\end{tabular}
\end{table}

\FloatBarrier

%-------------------------------------------------------------------------------
\subsection{Identifizierung eines aktiven Nuklids per Spektrum}
\label{sec:Nuklidbestimmung}
Im letzten Schritt der Auswertung geht es darum, ein aktives Nuklid anhand eines
Spektrums zu identifizieren. Dabei besteht keine Entscheidungsmöglichkeit zwischen
zwei gegebenen, wie in dem vorherigen Schritt.
Das in Abbildung \ref{plt:unbekannt} dargestellte Spektrum wurde dazu genutzt, die Peaks herauszusuchen.
Anschließend wurden deren Höhe und Energie bestimmt. Die daraus
resultierenden Werte sind in Tabelle \ref{tab:last} zu finden.
\begin{figure}[htb]
  \centering
  \includegraphics[width=0.8\textwidth]{build/unbekannt.pdf}
  \caption{Darstellung eines Spektrums einer unbekannten Quelle.}
  \label{plt:unbekannt}
\end{figure}
\input{build/tables/last.tex}
\FloatBarrier
Die Peaks wurden wieder  \textit{find.peaks} bestimmt und dann Mithilfe von der
Datenbank aus Referenz \cite{referenz} dem Isotop $^{60}\text{Co}$ identifiziert.
Bei diesem Isotop trifft der Peak bei \SI{1172.20}{\kilo\electronvolt} in
\SI{99.85}{\percent} und der Peak bei \SI{1332.75}{\kilo\electronvolt} in
\SI{99.98}{\percent} der Fälle auf.
