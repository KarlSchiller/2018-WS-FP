\newpage
\section{Auswertung}
\label{sec:Auswertung}

%-------------------------------------------------------------------------------
\subsection{Energiekallibrierung}
\label{sec:Energiekallibrierung}

Für eine Energiekalibrierung wurde das Spektrum der \ce{^{152}Eu}-Quelle
aufgezeichnet. Dies ist in Abbildung \ref{plt:Eu-Spektrum} zu sehen. Die zu
erkennenen Peaks wurden dem erwartetem Europium-Spektrum zugeordnet.
\begin{figure}[htb]
  \begin{subfigure}{0.49\textwidth}
    \centering
    \includegraphics[width=\textwidth]{build/orginal_Eu.pdf}
  \end{subfigure}
  \begin{subfigure}{0.49\textwidth}
    \centering
    \includegraphics[width=\textwidth]{build/orginal_Eu_log.pdf}
  \end{subfigure}
  \caption{Vom Germanium Detektor aufgenommenes Spektrum der Europium Quelle mit
  dekadischer und logarithmischer y-Achse.}
  \label{plt:Eu-Spektrum}
\end{figure}
Anhand der Energien und Bin-Indizes wurde eine lineare Ausgleichsrechnung, in
Abbildung \ref{plt:eichung} zu erkennen, mit Hilfe der Formel
\begin{align*}
  E = m \cdot x + b
\end{align*}
durchgeführt.
\begin{figure}[htb]
  \centering
  \includegraphics[width=0.8\textwidth]{build/kalibation.pdf}
  \caption{Zugeordnete Energien aufgetragen gegen die Bin-Indizes zur
  Kalibrierung des Detektors für die beobachteten Peaks (blau). Zudem ist eine
  lineare Regression der Werte (rot) aufgetragen.}
  \label{plt:eichung}
\end{figure}
Dabei entspricht $x$ der Bin-Indizes. Für die Parameter $m$ und
$b$ ergeben sich:
\begin{align*}
	m &= \SI{0.4204(69)}{\kilo\electronvolt} \\
  b &= \SI{-41.25(1428)}{\kilo\electronvolt}
\end{align*}

\begin{table}
	\centering
	\caption{Gegebene Werte zur Kalibrierung des Germanium-Detektors \cite{anleitung}.}
	\label{tab:anleitung_eu}
	\begin{tabular}{
		S[table-format=1.0]
		S[table-format=2.1]
		S[table-format=1.0]
		}
	\toprule
		{Energie $E\;/\;\si{\si{\kilo\electronvolt}$}} &
		{Bin-Index $i$} &
		{Emis.-Wahr. W} \\
	\midrule
		 122 &  28.6 &  378 \\
		 245 &  7.6 &  691 \\
		 296 &  0.4 &  742 \\
		 344 &  26.5 &  861 \\
		 411 &  2.2 &  1108 \\
		 444 &  3.1 &  1220 \\
		 678 &  2.0 &  1716 \\
		 689 &  0.9 &  1790 \\
		 779 &  12.9 &  1939 \\
		 867 &  4.2 &  2185 \\
		 964 &  14.6 &  2400 \\
		 1005 &  0.6 &  2502 \\
		 1086 &  10.2 &  2703 \\
		 1112 &  13.6 &  2765 \\
		 1299 &  1.6 &  3014 \\
		 1408 &  21.0 &  3501 \\
	\bottomrule
	\end{tabular}
\end{table}
\FloatBarrier

%-------------------------------------------------------------------------------
\subsection{Effizienzmessung des Detektors}
\label{sec:Effizienzmessung}

Zur Berechnung der Detektoreffizienz wurde Formel xy verwendet.  % TODO: Formel einfügen
Dafür wurde zuerst der Raumwinkel anhand Formel xy berechnet. Dabei  % TODO: Formel einfügen
ergibt mit $a = \SI{22.3(10)}{\milli\meter}$ und $r = \SI{22.5(1)}{\milli\meter}$
sich:
\begin{align*}
  \frac{\Omega}{4\pi} = \num{0.01558(34)}
\end{align*}
Dabei wurde $r$ so gewählt, dass der wahrscheinlichste Wechselwirkungspunkt
\SI{1.5}{\centi\meter} innerhalb des Germaniums liegt und der Abstandshalter
zwischen Probe und Detektor \SI{7.31}{\centi\meter} lang ist.
Die Aktivität wurde aus dem Wissen errechnet, dass am 01.10.2000 die Aktivität
der Europium-Probe bei $A = \SI{4130(60)}{\becquerel}$ beträgt. Mithilfe der
Halbwertszeit $t_{\sfrac{1}{2}} = \SI{4943(5)}{\day}$ folgt für den Messtag:
\begin{align*}
    A_\text{Messtag} = A_0 \exp{-\frac{\ln{2} \Delta t}{t_\frac{1}{2}}}=\SI{1633(24)}{\becquerel}
\end{align*}
Der Peakinhalt wurde mit einem Fit durch eine Gaußfunktion der Form
\begin{align*}
	f\left(x\right) = h\cdot \exp{\frac{(x-\mu)^2}{2\sigma^2}} + a
\end{align*}
für jede Energie durchgeführt. Hierbei beschreibt $h$ die Höhe des Peaks, $\mu$
den Mittelwert (um Bin-Indizes der Peaks zu korrigieren), $\sigma$ die
Standardabweichnug und $a$ einen Parameter zur Untergrundberücksichtigung
bezeichnet. Die Parameter der Fits für jedes Bin sind in Tabelle
\ref{tab:gauss_parameter} nachzulesen.

\begin{table}
	\centering
	\caption{Parameter des durchgeführten Gauss-Fits pro Bin. Dabei ist $\mu$ der Mittelwert, $\sigma$ die Standardabweichnug, $h$ die Höhe und a der Energieoffset.}
	\label{tab:gauss_parameter}
	\begin{tabular}{
		S[table-format=1.2] @{${}\pm{}$} S[table-format=1.2]
		S[table-format=1.2] @{${}\pm{}$} S[table-format=1.2]
		S[table-format=1.2] @{${}\pm{}$} S[table-format=1.2]
		S[table-format=1.2] @{${}\pm{}$} S[table-format=1.2]
		}
	\toprule
		\multicolumn{2}{c}{$a$} &
		\multicolumn{2}{c}{$h_i$} &
		\multicolumn{2}{c}{$\mu_i$} &
		\multicolumn{2}{c}{$\sigma_i$} \\
	\midrule
		 68.14 &  1.81 &  18.53 &  2.59 &  384.09 &  1.61 &  11.59 &  2.24 \\
		 28.53 &  0.62 &  36.42 &  209088.64 &  691.33 &  608.71 &  0.26 &  477.23 \\
		 25.80 &  0.56 &  30.13 &  3.79 &  741.15 &  0.20 &  1.39 &  0.20 \\
		 21.02 &  1.00 &  1367.21 &  6.25 &  860.92 &  0.01 &  1.58 &  0.01 \\
		 16.76 &  0.57 &  118.32 &  3.36 &  1108.16 &  0.06 &  1.75 &  0.06 \\
		 16.54 &  0.50 &  7.93 &  3.03 &  1219.87 &  0.74 &  1.69 &  0.76 \\
		 15.21 &  0.42 &  25.36 &  3.66 &  1715.98 &  0.14 &  0.81 &  0.13 \\
		 14.05 &  0.41 &  13.86 &  3.53 &  1789.76 &  0.26 &  0.89 &  0.26 \\
		 12.66 &  0.53 &  180.12 &  2.63 &  1939.14 &  0.04 &  2.39 &  0.04 \\
		 27.24 &  6.78 & -19.01 &  6.23 &  2205.59 &  5.66 &  26.95 &  12.42 \\
		 6.42 &  0.66 &  148.74 &  2.84 &  2398.69 &  0.07 &  3.00 &  0.07 \\
		 5.18 &  0.29 &  7.98 &  1.39 &  2500.53 &  0.50 &  2.48 &  0.51 \\
		 6.44 &  0.62 &  70.59 &  2.39 &  2701.61 &  0.14 &  3.64 &  0.15 \\
		 5.67 &  0.55 &  104.52 &  2.30 &  2766.12 &  0.08 &  3.22 &  0.08 \\
		 3.66 &  0.29 &  9.67 &  1.40 &  3015.25 &  0.42 &  2.57 &  0.44 \\
		 0.69 &  0.45 &  107.38 &  1.70 &  3500.68 &  0.07 &  3.81 &  0.07 \\
	\bottomrule
	\end{tabular}
\end{table}
\FloatBarrier

Aus diesen ergibt sich der Peakinhalt $Z_i$ des Peaks $i$ durch Abintegration
über eine Gaußkurve:
\begin{align*}
  Z_i = \sqrt{2\pi} h_i \sigma_i
\end{align*}

Nun kann aus den gerade berechneten Werten, in Tabelle \ref{tab:det_eff}
aufgelistet, dazu verwenden einen Fit der Form
\begin{align*}
  Q(E) = a \cdot (E - b)^e + c
\end{align*}
durchzuführen. Dabei wurden nur solche Energien betrachtet, die über
\SI{150}{\kilo\electronvolt} liegen. Daraus ergeben sich die Abbildung
\ref{plt:eff} die folgenden Parameter:
\begin{align*}
  a &= -\SI{0.1(23)}{\kilo\electronvolt\per\becquerel} \\
  b &= \SI{2(12)}{\kilo\electronvolt} \\
  c &= \SI{6(20)}{\raiseto{-1}\becquerel} \\
  e &= \num{0.5(20)}
\end{align*}

\begin{figure}[htb]
  \centering
  \includegraphics[width=0.8\textwidth]{build/efficiency.pdf}
  \caption{Fit zur Effizienzbestimmung des Detektors anhand der zuvor
  berechneten Werten der Efizienz anhand der Energien.}
  \label{plt:eff}
\end{figure}

\begin{table}
	\centering
	\caption{Peakhöhe, Energie und Detektoreffizenz als Ergebnis des Gaußfits.}
	\label{tab:det_eff}
	\begin{tabular}{
		S[table-format=1.2] @{${}\pm{}$} S[table-format=1.2]
		S[table-format=1.2]
		S[table-format=1.2] @{${}\pm{}$} S[table-format=1.2]
		}
	\toprule
		\multicolumn{2}{c}{$Z_i$} &
		{E_i} &
		\multicolumn{2}{c}{$Q \ \si{becquerel}$} \\
	\midrule
		 538.41 &  128.48 &  120.24+/-0.68 &  0.74 &  0.18 \\
		 24.06 &  144843.44 &  249.41+/-255.92 &  0.12 &  749.00 \\
		 104.94 &  20.29 &  270.36+/-0.08 &  10.31 &  2.00 \\
		 5423.10 &  38.12 &  320.71+/-0.00 &  8.04 &  0.13 \\
		 519.47 &  22.69 &  424.66+/-0.02 &  9.28 &  0.43 \\
		 33.65 &  19.80 &  471.63+/-0.31 &  0.43 &  0.25 \\
		 51.61 &  11.13 &  680.21+/-0.06 &  1.01 &  0.22 \\
		 30.91 &  12.06 &  711.23+/-0.11 &  1.35 &  0.53 \\
		 1079.54 &  24.38 &  774.04+/-0.02 &  3.29 &  0.09 \\
		-1284.03 &  725.94 &  886.06+/-2.38 & -12.02 &  6.80 \\
		 1118.21 &  33.15 &  967.25+/-0.03 &  3.01 &  0.10 \\
		 49.64 &  13.42 &  1010.07+/-0.21 &  3.25 &  0.88 \\
		 644.04 &  33.97 &  1094.61+/-0.06 &  2.48 &  0.14 \\
		 844.48 &  28.90 &  1121.73+/-0.03 &  2.44 &  0.09 \\
		 62.29 &  13.92 &  1226.48+/-0.18 &  1.53 &  0.34 \\
		 1024.99 &  25.27 &  1430.57+/-0.03 &  1.92 &  0.05 \\
	\bottomrule
	\end{tabular}
\end{table}
\FloatBarrier

%-------------------------------------------------------------------------------
\subsection{Bestimmung der Detektoreigenschaften}
\label{sec:Detektoreigenschaften}
Für diesen Teil der Auswertung wurde das Spektrum eines Cäsium-Strahlers, in
Abbildung \ref{plt:Cs} zu sehen, aufgenommen.

\begin{figure}
  \begin{subfigure}{0.49\textwidth}
    \centering
    \includegraphics[width=\textwidth]{build/spektrum_Cs.pdf}
  \end{subfigure}
  \begin{subfigure}{0.49\textwidth}
    \centering
    \includegraphics[width=\textwidth]{build/spektrum_Cs_log.pdf}
  \end{subfigure}
  \caption{Vom Detektor aufgenomenes Spektrum der Cäsium-Quelle, dargestellt
  mit dekadischer und logarithmischer y-Achse.}
  \label{plt:Cs}
\end{figure}
Hier wurden die charakteristischen
Peaks des Spektrums (Rückstreu- und Vollenergiepeak, sowie die Comptonkante)
identifiziert. Die Zuordnung dieser sind in \ref{tab:Cs_char} nachzulesen. Der
Theoriwert des Vollenergiepeaks liegt bei \SI{661.59}{\kilo\electronvolt}
\cite{theorie}.
\begin{table}[htb]
	\centering
  \begin{tabular}{c
    S[table-format=4.0]
    S[table-format=3.2]}
    \toprule
    {} & {Index $i$} & {Energie $E_i$ / keV} \\
    \midrule
    Rückstreupeak & 511 & 173,59 \\
    Compton-Kante & 1174 & 452,63 \\
    Vollenergiepeak & 1648 & 651,63 \\
    \bottomrule
  \end{tabular}
  \caption{Charakteristische Peaks des Cs-Strahlers.}
  \label{tab:Cs_char}
\end{table}

Eine Berechnung der Theoriwerte für Rückstreupeak und Comptonkante durch die
Energie des Vollenergiepeaks $E_{\text{Voll}}$ normiert auf $m_0\:c^2$ ergibt:
\begin{align*}
  E_\text{Compton, Theo} &= \frac{2\epsilon}{1+2\epsilon}\cdot E_\text{Voll} = \SI{1136.32}{\kilo\electronvolt} \\
  E_\text{Rück, Theo} &= \frac{1}{1+2\epsilon}\cdot E_\text{Voll} = \SI{659.88}{\kilo\electronvolt}
\end{align*}
Diese Werte weichen wie folgt von den gemessenen Werten ab:
\begin{align*}
  \Delta E_\text{Compton} &= \SI{60.19}{\percent} \\
  \Delta E_\text{Rück} &= \SI{73.69}{\percent}
\end{align*}

Um zu überprüfen, ob der Vollenergiepeak eine Gaussform annimmt, wird eine
rechte (r) und eine linke (l) Flanke mithilfe einer linearen Regression
angenommen. Daraus ergeben sich die Parameter:
\begin{align*}
  a_\text{l} &= \num{0.0042(7)} \quad a_\text{r} = \num{-0.0039(6)} \\
  b_\text{l} &= \num{1640.6(6)} \quad b_\text{r} = \num{1655.2(6)}
\end{align*}
Durch Hinzunahme der Formel
\begin{align*}
	\Delta_\text{x} = m_\text{r}xh + b_\text{r} - \left(m_\text{l}xh + b_\text{l}\right)\ \text{mit}\ h \in \left\{\num{0.1}; \num{0.5}\right\}
\end{align*}
kann die Halbwerts- und die Zehntelbreite bestimmt werden. Daraus ergab sich:
\begin{align*}
  \Delta_{1/2} &= \SI{-38.3(5)}{\kilo\electronvolt} \\
	\num{1.823} \cdot \Delta_{1/2} &= \SI{-35.9(9)}{\kilo\electronvolt} \\
  \Delta_{1/10} &= \SI{-35.7(4)}{\kilo\electronvolt}
\end{align*}
Wird nun das Verhältnis zwischen den beiden unteren asdf  % TODO
betrachtet, so erhält man hier eine Abweichung zum ermittelten Wert von
$\Delta_{1/10}$ von \SI{5}{\percent}.

Im letzten Schritt dieses Auswertungsteils wird der Inhalt des Compton-Kontinuums
und des Vollenergiepeaks mit der Absorptionswahrscheilichkeit des Compton- und
Photoeffektes verglichen. Dafür wird die Formel
\begin{align*}
	p = 1 - \exp\left(-\mu l\right)
\end{align*}
mit der Länge des Detektors $l = \SI{3.9}{\centi\meter}$ und dem
Absorptionskoeffizienten $\mu$ verwendet. Dabei ergeben sich die
Absorptionskoeffizienten aus Abbildung \ref{plt:Cs_abs} zu:
\begin{align*}
	\mu_\text{Compton} &= \SI{111}{\per\centi\meter} & p_\text{Compton} &= \SI{111}{\percent} \\
	\mu_\text{Photo} &= \SI{111}{\per\centi\meter} & p_\text{Photo} &= \SI{111}{\percent}
\end{align*}

\begin{figure}
  \begin{subfigure}{0.49\textwidth}
    \centering
    \includegraphics[width=\textwidth]{build/Cs.pdf}
  \end{subfigure}
  \begin{subfigure}{0.49\textwidth}
    \centering
    \includegraphics[width=\textwidth]{build/Cs_log.pdf}
  \end{subfigure}
  \caption{Vollenergiepeak und Compton-Kontinuum des Cs-Spektrums.}
  \label{plt:Cs_abs}
\end{figure}

\subsection{Aktivitätsbestimmung anhand eines Spektrums}
\label{sec:Aktivitätsbestimmung}

\subsection{Identifizierung eines aktiven Nuklids per Spektrum}
\label{sec:Nuklidbestimmung}

