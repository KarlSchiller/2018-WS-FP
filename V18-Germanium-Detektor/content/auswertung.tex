\newpage
\section{Auswertung}
\label{sec:Auswertung}

\subsection{Energiekallibrierung}
\label{sec:Energiekallibrierung}
Für eine Energiekalibrierung wurde das Spektrum der ${152}^$Eu Quelle
aufgezeichnet. Dies ist in Abbildung \ref{plt:Eu-Spektrum} zu sehen. Die zu
erkennenen Peaks dem erwartetem Europium-Spektrum zugeordnet.

\begin{figure}[htb]
  \centering
  \includegraphics[width=0.8\textwidth]{build/Eu-gaugespektrum.pdf}
  \caption{Aufgenommenes ${152}^$Eu-Spektrum abgebildet auf die von Detektor aufgenommenen Ausschläge.}
  \label{plt:Eu-Spektrum}
\end{figure}

Anhand der Energien und Bin-Indizes wurde eine lineare Ausgleichsrechnung, in
Abbildung \ref{plt:Referenz} zu erkennen, mit Hilfe der Formel
\begin{equation}
  E = m \cdot x + b
\end{equation}
durchgeführt. Dabei entspricht $x$ der Bin-Indizes. Für die Parameter $m$ und
$b$ ergeben sich:
\begin{equation}
  m &= \SI{2.369(39)}{\kilo\electronvolt} \\
  b &= \SI{104.764(32408)}{\kilo\electronvolt}
\end{equation}

\begin{figure}[htb]
  \centering
  \includegraphics{build/Referenz.pdf}
  \caption{Energiekalibrierung am Eu-Spektrum mit linearer Ausgleichsrechnung.}
  \label{plt:Referenz}
\end{figure}

\subsection{Effizienzmessung des Detektors}
\label{sec:Effizienzmessung}
Zur Berechnung der Detektoreffizienz wurde nach Formel \eqref{} durchgeführt.
Dafür wurde zuerst der Raumwinkel anhand Formel \eqref{} berechnet. Dabei
ergibt mit $a=\SI{22.3(10){\milli\meter}}$ und $r=\SI{22.5(1)}{\milli\meter}$
sich:
\begin{equation}
  \frac{\Omega}{4\pi} = \Si{0.14876(8)}
\end{equation}

\subsection{Bestimmung der Detektoreigenshaften}
\label{sec:Detektoreigenschaften}

\subsection{Aktivitätsbestimmung anhand eines Spektrums}
\label{sec:Aktivitätsbestimmung}

\subsection{Identifizierung eines aktiven Nuklids per Spektrum}
\label{sec:Nuklidbestimmung}

% \subsection{Unterkapiel}
% \label{sec:Unterkapitel}

% \begin{figure}
%   \centering
%   \includegraphics[width=\textwidth]{Plot.pdf}
%   \caption{Bildunterschrift}
%   \label{fig:Plot1}
% \end{figure}
