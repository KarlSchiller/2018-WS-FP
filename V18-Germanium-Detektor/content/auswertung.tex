\newpage
\section{Auswertung}
\label{sec:Auswertung}

\subsection{Energiekallibrierung}
\label{sec:Energiekallibrierung}
Für eine Energiekalibrierung wurde das Spektrum der ${152}^$Eu Quelle
aufgezeichnet. Dies ist in Abbildung \ref{plt:Eu-Spektrum} zu sehen. Die zu
erkennenen Peaks dem erwartetem Europium-Spektrum zugeordnet.

\begin{figure}[htb]
  \begin{subfigure}{0.5\textwidth}
    \centering
    \includegraphics[width=\textwidth]{build/orginal_Eu.pdf}
  \end{subfigure}
  \begin{subfigure}{0.5\textwidth}
    \centering
    \includegraphics[width=\textwidth]{build/orginal_Eu_log.pdf}
  \end{subfigure}
  \caption{Vom Germanium Detektor aufgenommenes Spektrum der Europium Quelle mit
  dekadischer und logarithmischer y-Achse.}
  \label{plt:Eu-Spektrum}
\end{figure}

Anhand der Energien und Bin-Indizes wurde eine lineare Ausgleichsrechnung, in
Abbildung \ref{plt:eichung} zu erkennen, mit Hilfe der Formel
\begin{equation}
  E = m \cdot x + b
\end{equation}
durchgeführt.

\begin{figure}[htb]
  \centering
  \includegraphics[width=0.8\textwidth]{build/kalibation.pdf}
  \caption{Lineare Regression der zugeordneten Energien, aufgetragen gegen die jeweiligen Bin Indizes.}
  \label{plt:eichung}
\end{figure}

Dabei entspricht $x$ der Bin-Indizes. Für die Parameter $m$ und
$b$ ergeben sich:
\begin{equation}
  m &= \SI{0.4204(69)}{\kilo\electronvolt} \\
  b &= \SI{-41.25(1428)}{\kilo\electronvolt}
\end{equation}



\subsection{Effizienzmessung des Detektors}
\label{sec:Effizienzmessung}
Zur Berechnung der Detektoreffizienz wurde nach Formel ## durchgeführt.
Dafür wurde zuerst der Raumwinkel anhand Formel ## berechnet. Dabei
ergibt mit $a=\SI{22.3(10){\milli\meter}}$ und $r=\SI{22.5(1)}{\milli\meter}$
sich:
\begin{equation}
  \frac{\Omega}{4\pi} = \Si{0.01558(34)}
\end{equation}
Dabei wurde $r$ so gewählt, dass der wahrscheinlichste Wechselwirkungspunkt
\SI{1.5}{\centi\meter} innerhalb des Germaniums liegt und der Abstandshalter
zwischen Probe und Detektor \SI{7.31}{\centi\meter} lang ist.
Die Aktivität wurde aus dem Wissen errechnet, dass am 01.10.2000 die Aktivität
der Europium-Probe bei $A=\SI{4130(60)}{\becquerel}$ beträgt. Mithilfe der
Halbwertszeit $t_\frac{1}{2}=SI{4943(5)}{\days}$ folgt für den Messtag:
\begin{equation}
    A_\text{Messtag}=A_0 \exp{-\frac{\ln{2} \Delta t}{t_\frac{1}{2}}}=\SI{1633(24)}{\becquerel}
\end{equation}
Der Peakinhalt wurde mit einem Fit durch eine Gaußfunktion der Form
\begin{equation}
  f(x) = h\cdot \exp{\frac{(x-\mu)^2}{2\sigma^2}} + a
\end{equation}
für jede Energie durchgeführt. Hierbei beschreibt $h$ die Höhe des Peaks, $\mu$
den Mittelwert (um Bin-Indizes der Peaks zu korrigieren), $\sigma$ die
Standardabweichnug und $a$ einen Parameter zur Untergrundberücksichtigung
bezeichnet.
Daraus ergibt sich der Peakinhalt $Z_i$ des Peaks $i$ durch Abintegration über
eine Gaußkurve:
\begin{equation}
  Z_i = \sqrt{2\pi} h_i \sigma_i
\end{equation}


\subsection{Bestimmung der Detektoreigenshaften}
\label{sec:Detektoreigenschaften}

\subsection{Aktivitätsbestimmung anhand eines Spektrums}
\label{sec:Aktivitätsbestimmung}

\subsection{Identifizierung eines aktiven Nuklids per Spektrum}
\label{sec:Nuklidbestimmung}

% \subsection{Unterkapiel}
% \label{sec:Unterkapitel}

% \begin{figure}
%   \centering
%   \includegraphics[width=\textwidth]{Plot.pdf}
%   \caption{Bildunterschrift}
%   \label{fig:Plot1}
% \end{figure}
