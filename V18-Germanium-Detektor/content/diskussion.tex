\section{Diskussion}
\label{sec:Diskussion}
Die Energiekalibrierung ist wie in Abbildung \ref{plt:eichung}
zu sehen ohne nennenswerte Unsicherheiten durchgeführt worden. Dies spricht
dafür, dass die Zuordnung der Peaks zu den entsprechenden Energien korrekt ist.

Bei Berechnung der Aktivität am Messtag und des Raumwinkels wurden bis auf kleine
Unsicherheiten realistische Werte erhalten. Der Fit mithilfe einer gaußförmigen
Funktion ergab durchweg konsistente Parameter. Dabei ist die Unsicherheit auf
den zweiten Wert in Tabelle \ref{tab:gauss_parameter} und die dadurch
berechenten Werten in Tabelle \ref{tab:gauss_parameter} sehr hoch, was auf eine
nicht korrekte Zuordnung des entsprechenden Peaks hinweisen könnte. Ebenso ist
% TODO: Zuordnung überprüft? Zu zweit besprechen...
diese Begründung für den negativen Wert in dem jeweiligen Tabellen
anzuführen.

In Abbildung \ref{plt:Cs_abs} zur Untersuchung der Detektoreigenschaften ist erkennbar,
dass der Vollenergiepeak wesentlich höher liegt als
das Compton-Kontinuum. Dieser Effekt ergibt sich trotz der geringeren
Absorptionswahrscheinlichkeit dadurch, dass der Vollenergiepeak nicht durch den
Photoeffekt alleine hervorgerufen wird, sondern auch durch Elektronen, die ihre
Energie mehrfach durch den Comptoneffekt abgeben und dann dann erst durch den
Photoeffekt ihre Energie vollständig verlieren. Durch diesen Prozess resultiert
ein höherer Extinktionskoeffizient als durch den Photoeffekt allein.

Die Bestimmung des Nuklids von Barium oder Antimon zeigt bei Betrachtung der
berechneten Aktivität und der geringen Unsicherheiten der Gauß-Fit Parameters
eine ausreichend gute Zuordnung der Peaks zu den Energien. Bei der Berechnung
der Aktivität konnten zwei Werte nicht berechnet werden, da sie die Bedingung
$E_\gamma > \SI{150}{\kilo\electronvolt}$ nicht erfüllen. 

Beim letzten Schritt der Auswertung, dem Identifizieren eines aktiven Nuklids
anhand eines Spektrums, ist die Betrachtung der Emissionslinien in diesem Fall
einfach gehalten, da nur zwei Energien mit deutlich erhöhter Zählrate auftreten.
Dabei ist die genaue Unterscheidung zwischen Wismut und Titan nicht möglich, da
deren Emissionslinien sehr nah beieinander liegen.
