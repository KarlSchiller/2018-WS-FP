\section{Diskussion}
\label{sec:Diskussion}
Die Energiekalibrierung ist wie in Abbildung \ref{plt:eichung} durchgeführt worden.
Da keine der Emissionslinien von der Geraden abweicht, spricht dies für eine gute
Kalibration.

Bei Berechnung der Aktivität am Messtag und des Raumwinkels wurden kleine Unsicherheiten
in den gemessenen Größen angenommen, da bei Abmessungen von Längen und Zeitintervallen
immer eine gewisse Unsicherheit aufgrund von elektrischen oder mechanischen Messgeräten
auftreten. Zudem ist die Berechnung des Raumwinkels nur eine Näherung für eine Messung,
die in weiter Ferne des Strahlers vorgenommen wird und somit eine Punktquelle darstellt.
Diese Bedingung ist hier in der Allgemeinheit nicht anzuhemen, da sich die Probe nur
ungefähr \SI{10}{\centi\meter} von dem Detektor entfernt aufhält.
Der Fit mithilfe einer Gauß-Funktion ergab durchweg konsistente Parameter. Dabei
ist der Fitparameter $h_i$ der ersten und dritten Emissionslinie wesentlich höher
als die der anderen Linien. dies resultiert aus der hohe Emissionswahrscheinlichkeit
eben dieser Linien.

Eine Berechnung der Detektoreffizienzen aus Abbildung \ref{plt:eff} zeigt,
dass die berechneten Werte der Form der gefitteten Verteilung folgen. Die Effizienz
des Germaniumdetektors somit von kleinen zu immer größer werdenden Energien geringer
wird.

Die während der Bestimmung der Detektoreigenschaften aufgenommenen Energiewerte des
Rückstreupeaks und der Compton-Kante mithilfe der Caesium-Quelle weichen die gemessenen
von den theoretischen Werten ab. Für den
Rückstreupeak bertägt die Abweichung \SI{6.2}{\percent}. Bei der Compton-Kante beträgt
die Abweichung \SI{5.4}{\percent}. Die Abweichungen befinden sich in der gleichen
Größenordnung, was dafür spricht, dass die Abweichungen aus einer zu geringen Statistik
kommen und mit einer höheren Messzeit weiter zu verringern wären.

Unsicherheiten bei er Bstimmung der Halbwerts- und Zehntelbreiten folgen aus dem
Ablesen dieser aus der Abbildung \ref{plt:halb}. Trotzdem zeigen die bestimmten Werte
eine Übereinstimmung mit dem angenommenen Verhältnis zwischen Halbwerts- und
Zehntelbreite infolge einer Gaußverteilung. Daher ist diese Annahme, anstatt einer Poison-
eine gaußverteilung anzunehmen, als gerechtfertigt zu interpretieren.

Das Ablesen der Absorptionskoeffizienten aus Abbildung \ref{fig:Wirkungsquerschnitte-gesamt}
führt zu Unsicherheiten. Die daraus berechneten Absorptionswahrscheinlichkeiten
sind von unterschiedlicher Größenordnungen. Dies folgt daraus, dass der Photoeffekt
erst ab hohen Enerigen dominant wird und somit eine höhere Anzahl von Photonen vorhanden
sind, um in dem Detektor eine Paarerzeugung durchführen zu können. Bei den hier
vorliegen Energien (\SIrange{0}{700}{\kilo\electronvolt}) dominiert der Compton-Effekt,
was sich somit in der Absorptionswahrscheinlichkeit wiederspiegelt.

In Abbildung \ref{plt:Cs_abs} zur Untersuchung der Detektoreigenschaften ist erkennbar,
dass der Vollenergiepeak wesentlich höher liegt als
das Compton-Kontinuum. Dieser Effekt ergibt sich trotz der geringeren
Absorptionswahrscheinlichkeit dadurch, dass der Vollenergiepeak nicht durch den
Photoeffekt alleine hervorgerufen wird, sondern auch durch Elektronen, die ihre
Energie mehrfach durch den Comptoneffekt abgeben und dann dann erst durch den
Photoeffekt ihre Energie vollständig verlieren. Durch diesen Prozess resultiert
ein höherer Extinktionskoeffizient als durch den Photoeffekt allein.

Das aufgenommene Spektrum in Abbildung \ref{plt:bar} passt anhand der Peaks am besten
zu einer Barium-Quelle. Dabei wurden die aufgenommenen Peaks mit den angegebenen Peaks
aus der Referenz \cite{referenz} verglichen. Einige der angegebenen Peaks konnten
dabei aufgrund einer geringen Auftrittswahrscheinlichkeit nicht gefunden werden.
In Tabelle \ref{tab:aktivitaet_ba} konnte die Aktivität des ersten Peaks nicht
berechnet werden, da eine Berechnung der Detektoreffizienz wie in Abschnitt \ref{sec:Effizienzmessung}
einen verschwindenen Wert liefert. Die zur Berechnung der Aktivität beitragenden
Emissionslinien zeigen ein konsistentes Bild, daher konnte die gemittelte Aktivität
trotzdem berechnet werden. Dies würde sich durch die Aufnahme weiterer Emissionslinien
für weitere Konsistenz in des Ergebnisses zeigen.

Beim letzten Schritt der Auswertung, dem Identifizieren eines aktiven Nuklids
anhand eines Spektrums, ist die die Identifizierung der Emissionslinien mit 2
deutlich hervorstehenden Linien simpel gehalten. Referenz \cite{referenz} ergab
nur das $^{60}Co$-Isotop, welches ein häufig vorkommendes und anhand deiner 2-Linien-Struktur
besonders gut nachzuweisen ist. Diese Auffassung ist konnsistent mit den hier
präsentierten Ergebnissen.
