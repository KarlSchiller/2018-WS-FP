\section{Diskussion}
\label{sec:Diskussion}
Die Durchführung der Energiekalibrierung ist wie in Abbildung \ref{plt:eichung}
zu sehen ohne nennenswerte UNsicherheiten durchgeführt worden. Dies spricht
dafür, dass die Zuordnung der Peaks zu den entsprechenden Energien korrekt ist.

Bei Berechnung der Aktivität am Messtag und des Raumwinkels wurden bis auf kleine
Unsicherheiten realistische Werte erhalten. Der Fit mithilfe einer gaußförmigen
Funktion ergab durchweg konsistente Parameter. Dabei ist die Unsicherheit auf
den zweiten Wert in Tabelle \ref{tab:gauss_parameter} und die dadurch
berechenten Werten in Tabelle \ref{tab:gauss_parameter} sehr hoch, was auf eine
nicht korrekte Zuornung des entsprechenden Peaks hinweisen könnte. Ebenso ist
diese Begründung in dem jeweils in den Tabellen auftretenden negativen Wert
anzuführen.

Die Bestimmung des Nuklid von Barium oder Antimon zeigt bei Betrachtung der
berechneten Aktivität und die geringen Unsicherheiten der Gauß-Fit Parameter
eine ausreichend gute Zuordnung der Peaks zu den Energien. Bei der Berechnung
der Aktivität konnten zwei Werte nicht berechnet werden, da sie die Bedingung
$E_\gamma > \SI{150}{\kilo\electronvolt}$ nicht erfüllen.


Beim letzten Schritt der Auswertung, dem identifizieren eines aktiven Nuklids
anhand eines Spektrums, ist die Betrachtung der Emissionslinien in diesem Fall
einfach gehalten, da nur zwei Energien mit deutlich erhöhter Zählrate auftreten.
Dabei ist de genaue Unterscheidung zwischen Wismut und Titan nicht möglich, da
deren Emissionslinien sehr nah bei einander liegen.
