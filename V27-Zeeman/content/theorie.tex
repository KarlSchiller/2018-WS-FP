\section{Zielsetzung}
\label{sec:Zielsetzung}
In diesem Versuch soll der Lande-Faktor $g_i$, die Aufspaltung $\Delta E$ der
Zeeman-Linien, sowie die Dispersionsgebiete und Auflösevermögen der
Lummer-Gehrcke-Platte bei den Wellenlängen $\lambda = \SI{643.8}{\nano\meter}$
und $\lambda = \SI{480.0}{\nano\meter}$.

\section{Theorie}
\label{sec:Theorie}
\subsection{Berechnung der magnetischen Momente}
\label{sec:Berechnung_magnetischer_Momente}
Die Hüllenelektronen besitzen zwei Drehimpulse, den Bahndrehimpuls ($\vec{l}$)
und den
Spin ($\vec{s}$) mit ihren Quantenzahlen
\begin{align}
  \mid\vec{l}\mid &= \sqrt{l\left(l+1\right)}\hbar \\
  \mid\vec{s}\mid &= \sqrt{s\left(s+1\right)}\hbar
  \label{eqn:Quantenzahlen}
\end{align}
Das magnetische Moment der Drehimpulseinheit \hbar
\begin{align}
  \mu_\text{B} := -\frac{1}{2}\text{e}_0 \frac{\hbar}{m_0}
  \label{eqn:bohr}
\end{align}
wird als Bohrsches Magneton bezeichnet. Daraus lassen sich die magnetischen
Momente der Drehimpulse ableiten zu
\begin{align}
  \vec{\mu}_l &= -\mu_\text{B} \frac{\vec{l}}{\hbar} = -\mu_\text{B} \sqrt{l\left(l+1 \right)} \vec{l}_\text{e}\\
  \vec{\mu}_s &= -\text{g}_\text{S} \cdot \frac{\mu_\text{B}}{\hbar} \vec{s} = -\text{g}_\text{S} \sqrt{s\left(s+1 \right)} \vec{s}_\text{e}.
  \label{eqn:momente}
\end{align}
Dabei sind $\vec{l}_\text{e}$ und $\vec{s}_\text{e}$ Einheitsvektoren in die Richtung
des jeweiligen Drehimpulses.
Der Lande-Faktor g$_\text{S}$ eines Elektrons hat dabei den Wert 2. Daraus resultiert die
magnetomechanische Anomalie des Elektrons. Dabei ist das magnetische Moment des
Spins ($s = \sfrac{\num{1}}{\num{2}}$) doppelt so groß wie für den Bahndrehimpuls
($l = \num{1}$).

\subsection{Wechselwirkung magnetischer Momente und Drehimpulse}
\label{sec:Wechselwirkungen}
Die Drehimpulse der Elektronen in einem Atom können auf verschiedene Arten
miteinander wechselwirken. Hier unterscheiden wir die Grenzfälle bei Atomen mit
niedriger und hoher Kernladungszahl.

Werden die Atome mit niedriger Kernladungszahl betrachtet, so können die Bahndrehimpulse
der Elektronen in den nicht abgeschlossenen Schalen vektoriell aneinander koppeln.
Den daraus entstehenden Impuls nennt man Gesamtdrehimpuls der Hülle
\begin{align}
  \vec{L} = \sum_i \vec{l}_i
  \label{eqn:gesamtbahndrehimpuls}
\end{align}
mit dem Betrag
\begin{align}
  \mid\vec{L}\mid = \sqrt{L\left(L+1 \right)} \hbar.
  \label{eqn:betraggesamtdrehipuls}
\end{align}
Der Gesamtdrehimpuls ist immer ganzzahlig, obwohl die einzelnen Bahndrehimpulse $\vec{l}_i$,
wie in Abbildung \ref{abb:drehimpulse} zu sehen, unterschiedlich koppeln können.
Je nach Gesamtdrehimpulsquantenzahl der Hülle L werden die Terme mit S, D, P und F
unterschieden (Drehimpulsdrehsymbole).
\begin{figure}[htb]
  \centering
  \includegraphics[width=0.8\textwidth]{images/V27.pdf}
  \caption{Vektorielle Kombination von zwei Drehimpulsen zu einem Gesamtdrehimpuls
  \cite{anleitung}. Hier werden $l_1 = 1$ und $l_2 = 2$ betrachtet.}
  \label{abb:drehimpulse}
\end{figure}
Auf die selbe Art und Weise koppeln die Spins zu einem Gesamtspin
\begin{align}
  \vec{S} = \sum_i \vec{s}_i.
  \label{eqn:gesamtspin}
\end{align}
Dabei kann dieser Gesamtspin Werte annehmen von $\frac{N}{2}$ über $\frac{N}{2}-1$
weiter bis $\num{0}$. $N$ beschreibt dabei die Gesamtzahl der Hüllenelektronen in
der unabgeschlossenen Schale.
Genauso besitzen Bahndrehimpuls der Hülle und der Gesamtspin magnetische Momente
der Form
\begin{align}
  \mid \vec{\mu}_\text{L} \mid &= \mu_\text{B} \sqrt{L\left(L+1 \right)} \\
  \mid \vec{\mu}_\text{S} \mid &= \text{g}_\text{S} \mu_\text{B} \sqrt{S\left(S+1 \right)}
  \label{eqn:gesamtmomente}
\end{align}
Weiter setzen sich Gesamtdrehimpuls und Gesamtspin zu einem Gesamtdrehimpuls
\begin{align}
  \vec{J} = \vec{L} + \vec{S}
  \label{eqn:gesamtdrehimpuls}
\end{align}
zusammen. Diese Koppelung wird als LS- oder Russell-Saunders-Kopplung bezeichnet.
Die zugehörige Quantenzahl kann je nach Spin ganz- oder halbzahlig sein. Als
Kurzschreibweise wird der Gesamtdrehimpuls beim Drehimpulssymbol als unterer Index
geschrieben. Als oberer Index wird die Multiplizität mit
\begin{align}
  M = 2\cdot S + 1
\end{align}
bezeichnet.

Als zweiter Grenzfall werden die Atome mit hoher Kernladungszahl betrachtet.
Da hier die Kopplung der Bahndrehimpulse und der Spins untereinander kleiner sind
als die der Drehimpulse miteinander, werden hier die Gesamtdrehimpulse definiert
über
\begin{align}
  \vec{j}_i = \vec{l}_i + \vec{s_i}.
  \label{eqn:gesamt}
\end{align}
Daraus ergibt sich im Gegensatz zum Grenzfall der geringen Kernladungszahlen nur
ein einziger Gesamtdrehimpuls
\begin{align}
  \vec{J} = \sum_i \vec{j}_i.
  \label{eqn:gesamtdrehimpuls}
\end{align}

\subsection{Aufspaltung der Energieniveus im homogenen Magnetfeld}
\label{sec:aufspaltung}
Um die Aufspaltung der Energieniveaus eines Atoms in einem homogenen Magnetfeld
beschreiben zu können, wird zuerst das magnetische Moment der Atomhülle berechnet.
Daraus ergibt sich
\begin{align}
  \vec{\mu} = \vec{\mu}_\text{L} + \vec{\mu}_\text{S}.
  \label{eqn:gesamtmoment}
\end{align}
Obwohl die Gleichungen \ref{eqn:gesamtdrehimpuls} und \ref{eqn:gesamtmoment}
auf gleiche Weise berechnet und beide Terme den Gesamtdrehimpuls der Hülle und
den Gesamtspin enthalten, fallen die Richtungen von $\vec{\mu}$ und $\vec{L}$
nicht zusammen. Daher wird das gesamtmagnetische Moment und in eine zu $\vec{J}$
parallele Komponente $\mu_{\parallel}$ und eine senkrechte Komponente $\mu_{\bot}$
aufgeteilt. Betrachtet man eine klassische Herangehensweise, so präzessiert das
$\mu$ um die gegebene Feldlinie und verschwindet somit im zeitlichen Mittel. Damit
verschwindet ebenso der quantenmechanische Erwartungswert. Wird der Betrag von
$\mu_\text{J}$
\begin{align}
  \mid \vec{\mu}_\text{J} \mid = \mu_\text{B} \text{g}_\text{J} \sqrt{J\left(J+i \right)}.
\end{align}
Hierbei stellt $\text{g}_\text{J}$ den Lande-Faktor des Atoms dar und wird durch
\begin{align}
  \text{g}_\text{J} := \frac{3J\left(J+1 \right) + S\left(S+1 \right) - L\left(L+1 \right)}{2J\left(J+1 \right)}
\end{align}
beschrieben.

\begin{figure}[htb]
  \centering
  \includegraphics[width=0.8\textwidth]{images/V27_1.pdf}
  \caption{Energieaufspaltung bei einem Gesamtdrehimpuls $J = \num{2}$ \cite{anleitung}.}
  \label{abb:aufspaltung}
\end{figure}

\subsection{Auswahlregeln}
\label{sec:Auswahlregeln}

\begin{figure}[htb]
  \centering
  \includegraphics[width=0.8\textwidth]{images/V27_2.pdf}
  \caption{Graphische Darstellung des Verhältnisses zwischen Kartesischen- und Polarkoordniaten \cite{anleitung}.}
  \label{abb:polar}
\end{figure}

\subsection{normaler-/annormaler Zeemann-Effekt}
\label{sec:Zeemann}
Bei ausgeschaltetem Magnetfeld und Vernachlässigung des Spins wird der Effekt der
Aufspaltung von Spektrallinien normaler Zeemann-Effekt genannt.

\begin{figure}[htb]
  \centering
  \includegraphics[width=0.8\textwidth]{images/V27_3.pdf}
  \caption{Linienaufspaltung der Spektrallinien beim normalen Zeemann-Effekt \cite{anleitung}.pdf}
  \label{abb:normal}
\end{figure}

\begin{figure}[htb]
  \centering
  \includegraphics[width=0.8\textwidth]{images/V27_4.pdf}
  \caption{Spektrallinienaufspaltung beim normalen Zeemann-Effekt \cite{anleitung}.}
  \label{abb:spektral}
\end{figure}

\begin{figure}[htb]
  \centering
  \includegraphics[width=0.8\textwidth]{images/V27_5.pdf}
  \caption{Linienaufspaltung der Spektrallinien beim annormalen Zeemann-Effekt am Beispiel eines Alkali-Dubletts \cite{anleitung}.}
  \label{abb:annormal}
\end{figure}
