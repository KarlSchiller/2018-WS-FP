\newpage
\section{Diskussion}
\label{sec:Diskussion}

Die Qualität der Ergebnisse ist im großen Maße abhängig von der Präzision
der Justage. Um diese zu verbessern, konnte ein Imbusschlüssel in die
Drehknöpfe gesteckt werden und ermöglichte einen größeren Hebel.
Trotzdem war die Laserdiode sehr empfindlich gegen kleine Änderung der
Hebel- bzw. Drehknopfposition.
Auch ein Stoß am Tisch konnte für eine Veränderung sorgen.
Aus diesem Grund ist nicht sicher gestellt, dass jeweils die best mögliche
Einstellung erreicht wurde.

Bei dem am Ende von Abschnitt \ref{sec:Rb-Fluoreszenz} beschriebenen Durchfahren
der Moden durch Drehen des SIDE Drehknopfes ließen sich nur zwei Absorptionslinien
darstellen. Dies ist ebenfalls auf Justageprobleme zurück zu führen.
Nach Abzug der Hintergrundintensität waren jedoch vier Absorptionslinien
erkennbar.
