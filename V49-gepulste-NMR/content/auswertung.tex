\newpage
\section{Auswertung}
\label{sec:Auswertung}

\subsection{Justage der Apparatur}
Um ein möglichst homogenes Magnetfeld erzeugen zu dürfen, wurden zunächst die
Shim-Parameter optimiert. Die somit eingestellten Parameter sind:
\begin{align*}
  x &= -\num{1.77}  \\
  y &= -\num{4.94}  \\
  z &= +\num{3.44}  \\
  z^2 &= \num{3.15} \\
\end{align*}
Neben den Shim-Parametern wurden die Lamor-Frequenz $\omega_\text{L}$, die
Pulslänge $\Delta t_\text{90}$ und die Referenzphase $\phi$ eingestellt auf:
\begin{align*}
  \omega_\text{L} &= \SI{21.71198}{\mega\hertz} \\
  \Delta t_\text{90} &= \SI{4.40}{\mu\second} \\
  \phi &= \SI{60}{\degree} \\
\end{align*}
Diese Parameter wurden bei den einzelnen Versuchsteilen nachgeprüft und
verändert. Die Nachjustierten Parameter sind jeweils angegeben.

\subsection{Bestimmung der lonitudinalen Relaxationszeit}
Zur Bestimmung der longitudinalen Relaxationszeit $T_1$ wurde die in die
Spule induzierte Spannung und der Pulsabstand $\tau$ zwischen \SI{180}{\degree}-
und \SI{90}{\degree}-Puls gemessen. Die aufgenommenen Daten sind in Tabelle
\ref{tab:T1} aufgeführt. Als Regression ist eine Formel der Form
\begin{align*}
  U(t) = U_0 \cdot \left(1- 2\text{e}^{-\frac{\tau}{T_1}}\right) + U_1
\end{align*}
verwendet worden. Da die Werte eine Asymmetrie aufweisen, und somit nicht von
$-U_0$ bis $U_=$ gehen, muss der Parameter $U_1$ hinzugefügt werden. Daraus
folgen die Paraeter
\begin{align*}
  U_0 &= -\SI{631(4)}{\milli\volt} \\
  T_1 &= \SI{15(4)}{\milli\second} \\
  U_1 &= \SI{2.149(41)}{\volt}
\end{align*}
In Abbildung \ref{plt:T1} sind Messwerte, sowie die Regression graphisch dargestellt.
\begin{figure}[htb]
  \centering
  \includegraphics{build/t1.pdf}
  \caption{Graphische Abbildung der Meswerte und der Regression der longitudinalen Relaxationszeit.}
  \label{plt:T1}
\end{figure}


\subsection{Bestimmung der transversalen Relaxationszeit}
Bei Bestimmung der transversalen Relaxationszeit $T_2$ werden zwei Methoden
betrachtet. Zuvor wurden die Shim-Parameter neu angepasst. Nun betragen die Werte:
\begin{align*}
  x &= -\num{1.44} \\
  y &= -\num{4.95} \\
  z &= +\num{3.43} \\
  z^2 &= +\num{3.06}
\end{align*}
Zudem beträgt die Lamor-Frequenz $\omega_\text{L} = \SI{21.71179}{\mega\hertz}$, während die
Referenzphase und die \SI{90}{\degree}-Pullslänge
\begin{align*}
  \phi &= \SI{60}{\degree} \\
  \Delta t_{90} &= \SI{4.40}{\micro\second}
\end{align*}
Im Weiteren werden die Meiboom-Gill-Methode, sowie die Carr-Purcell-Methode
einzeln betrachtet.
\newpage

\subsubsection{Meiboom-Gill-Methode}
Bei der Meiboom-Gil-Mehode wurde eine Periodendauer von \SI{10}{\second}
verwendet, da so keine Effekte der longitudinalen Relaxationszeit $T_1$ die
Messung beeinflussen kann. In Abbildung \ref{fig:MG} ist die Sequenz der Messung
aufgeführt.
\begin{figure}[htb]
  \centering
  \includegraphics[width=0.8\textwidth]{rohdaten/mg_2.png}
  \caption{Aufnahme der Meiboom-Gill-Methode mit eingestelltem Pulsabstand.}
  \label{fig:MG}
\end{figure}
Um die transversale Relaxatinoszeit $T_2$ zu berechnen werden die Minima mit der
Funktion $find\_peaks$ des Pythonpakets $scipy\.signal$ bestimmt und
sind in Tabelle \ref{tab:MG} aufgelistet. Eine Regression ist mit einer Formel
der Form
\begin{align*}
  U(t) = U_0 \cdot \text{e}^{-\frac{t}{T_2}}
\end{align*}
mit $t = 2\tau$ durchgeführt worden. Die daraus erhaltenen Parameter sind
\begin{align*}
  U_0 &= -\SI{588}{\milli\volt} \\
  T_2 &= \SI{1.657(17)}{\second}.
\end{align*}
Dieser Zusammenhang zwischen der induzierten Spannung und dem Zeitabstand der
Pulse ist in Abbildung \ref{plt:MG} zu erkennen.
\begin{figure}[htb]
  \centering
  \includegraphics{build/MG.pdf}
  \caption{Graphische Darstellung der Minima der $\SI{180}{\degree}$-Peaks.}
  \label{plt:MG}
\end{figure}
\FloatBarrier

\subsubsection{Carr-Purcell-Methode}
Für die Aufnahme der Cerr-Purcell-Methode wurden die Shim-Parameter auf der
folgenden Werte eingestellt:
\begin{align*}
  x &= -\num{0.84} \\
  y &= -\num{4.92} \\
  z &= +\num{3.72} \\
  z^2 &= +\num{2.84}
\end{align*}
Zudem wurde die Lamor-Frequenz auf $\omega_\text{L} = \SI{21.71223}{\mega\hertz}$
verwendet. Die aufgenommene Sequenz ist in Abbildung \ref{feg:CP} dargestellt.
\begin{figure}[htb]
  \centering
  \includegraphics[width=0.8\textwidth]{rohdaten/cp_2.png}
  \caption{Am Oszilloskop aufgenommene Sequenz der Carr-Purcell-Methode.}
  \label{fig:CP}
\end{figure}
Die aufgenomenen Daten können nicht zur Bestimmung der transversalen
Relaxationszeit verwendet werden, da bei jeder Pulsgebung eine Phasenverschiebung
auftritt. Nach wenigen Iterationen tritt somit der Effekt einer Oszillation auf,
wie in Abbildung \ref{fig:CP2} bei kleinerer Pulslänge des \SI{180}{\degree}-Pulses
zu erkennen ist.
\begin{figure}[htb]
  \centering
  \includegraphics[width=0.8\textwidth]{rohdaten/cp_3.png}
  \caption{Am Oszilloskop aufgenommene Carr-Purcell-Methode mit geringer
  \SI{180}{\degree}-Pulslänge}
  \label{fig:CP2}
\end{figure}
\FloatBarrier


\subsection{Bestimmung der Halbwertsbreite}
Zur Bestimmung der Halbwertsbreite $t_\sfrac{1}{2}$ eines Spin-Echos wird das
Maxima des Spin Echos bestimmt. Bei halber Höhe des Maximums wird die Differenz
der entsprechenden $\tau$ Werte berechnet. Graphisch ist die durch die Messung
bestimmte Halbwertsbreite in Abbildung \ref{plt:t12} durch die rote gestrichelte
Linie gekennzeichnet. Zur Referenz ist in \ref{fig:t12} das zugehörige Bild der Sequenz am Oszilloskop zu sehen. Dabei beträgt der Wert für die Halbwertsbreite
\begin{align*}
  t_\sfrac{1}{2} = \SI{}{\second}
\end{align*}
\begin{figure}[htb]
  \centering
  \includegraphics{build/halbwertsbreite.pdf}
  \caption{Messung eines Spin-Echos und Bestimmung der Halbwertsbreite.}
  \label{plt:t12}
\end{figure}
\begin{figure}[htb]
  \centering
  \includegraphics[width=0.7\textwidth]{rohdaten/halbwertsbreite.png}
  \caption{Sequenzbild des Spin-Echos zur Bestimmung der Halbwertsbreite.}
  \label{fig:t12}
\end{figure}
\FloatBarrier

\subsection{Bestimmung der Diffusionskonstante}
Die Diffusionskonstante wurde mittels Spin-Echo-Verfahren bestimmt. Dabei wurde
ein maimaler Gradient des Magnetfeldes in z-Richtung eingestellt. Dieser beträgt
$z = -\num{9.00}$. Durch diese Anpassung wird ein möglichst inhomogenes Magnetfeld
erzeugt und somit ein Diffusionseffekt messbar.
Um die Diffusionskonstante $D$ berechnen zu können, müssen zunächst der Faktor
$G$ berechnet werden. Dazu wird $G$ mittels der Formel
\begin{align*}
  G = \frac{4\cdot 2,2}{d \gamma t_\sfrac{1}{2}}
\end{align*}
mit $d = \SI{4.4}{\milli\meter}$ berechnet.
Daraus ergibt sich
\begin{align*}
  G = \SI{8.7e-5}{\frac{\tesla}{\meter}}
\end{align*}
Dazu wird der erste Echo-Puls bei $t = 2\tau$ für verschiedene Pulslängen
verwendet. Die Regression wird mit Hilfe
einer Funktion der Form von Gleichung \eqref{} durchgeführt. Verwendet wurden die
Messdaten aus Tabelle \ref{tab:dif}.

Als Fitparameter ergeben sich:
\begin{align*}
  U_0 &= \SI{628(6)}{\milli\volt} \\
  D &= \SI{0.00174(5)}{\frac{\meter}{\seond}}
\end{align*}

\subsection{Bestimmung der Viskosität}


\subsection{Bestimmung des Molekülradius}


Carr-Purcell diskutieren, warum quantitativ nicht auswertbar
Oszillationen bei höheren Zeiten: Winkeldelta addiert sich iwann
wieder auf 180°, damit wieder Signal
T1-Messung
Anstelle des Fits in der Anleitung (bei dem M0 quasi zufällig gewählt werden muss):
Nichtlinearer Fit an logarithmierte Messwerte
Fit der Form M0 * (1 - 2*exp(-t/tau)) + M1
mit Offset M1
Bitte Form noch einmal genauer überprüfen

% \subsection{Unterkapiel}
% \label{sec:Unterkapitel}

% \begin{figure}
%   \centering
%   \includegraphics[width=\textwidth]{Plot.pdf}
%   \caption{Bildunterschrift}
%   \label{fig:Plot1}
% \end{figure}
