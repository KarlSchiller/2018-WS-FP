\newpage
\section{Auswertung}
\label{sec:Auswertung}

\subsection{Justage der Apparatur}
Um ein möglichst homogenes Magnetfeld erzeugen zu dürfen, wurden zunächst die
Shim-Parameter optimiert. Die somit eingestellten Parameter sind:
\begin{align*}
  x &= -\num{1.77}  \\
  y &= -\num{4.94}  \\
  z &= +\num{3.44}  \\
  z^2 &= \num{3.15} \\
\end{align*}
Neben den Shim-Parametern wurden die Lamor-Frequenz $\omega_\text{L}$, die
Pulslänge $\Delta t_\text{90}$ und die Referenzphase $\phi$ eingestellt auf:
\begin{align*}
  \omega_\text{L} &= \SI{21.71}{\mega\hertz} \\
  \Delta t_\text{90} &= \SI{4.40}{\mu\second} \\
  \phi &= \SI{60}{\degree} \\
\end{align*}
Diese Parameter wurden bei den einzelnen Versuchsteilen nachgeprüft und
verändert. Die Nachjustierten Parameter sind jeweils angegeben.

\subsection{Bestimmung der lonitudinalen Relaxationszeit}
Zur Bestimmung der longitudinalen Relaxationszeit $T_1$ wurde die in die
Spule induzierte Spannung und der Pulsabstand $\tau$ zwischen \SI{180}{\degree}-
und \SI{90}{\degree}-Puls gemessen. Die aufgenommenen Daten sind in Tabelle
\ref{tab:T1} aufgeführt. Als Regression ist eine Formel der Form
\begin{align*}
  U(t) = U_0 \cdot \left(1- 2\text{e}^{-\frac{\tau}{T_1}}\right) + U_1
\end{align*}
verwendet worden. Da die Werte eine Asymmetrie aufweisen, muss der Parameter
$U_1$ hinzugefügt werden.

\subsection{Meiboom-Gill-Methode}

\subsection{Carr-Purcell-Methode}

\subsection{Bestimmung der Diffusionskonstante}

\subsection{Bestimmung der Halbwertsbreite}


Carr-Purcell diskutieren, warum quantitativ nicht auswertbar
Oszillationen bei höheren Zeiten: Winkeldelta addiert sich iwann
wieder auf 180°, damit wieder Signal

T1-Messung
Anstelle des Fits in der Anleitung (bei dem M0 quasi zufällig gewählt werden muss):
Nichtlinearer Fit an logarithmierte Messwerte
Fit der Form M0 * (1 - 2*exp(-t/tau)) + M1
mit Offset M1
Bitte Form noch einmal genauer überprüfen

% \subsection{Unterkapiel}
% \label{sec:Unterkapitel}

% \begin{figure}
%   \centering
%   \includegraphics[width=\textwidth]{Plot.pdf}
%   \caption{Bildunterschrift}
%   \label{fig:Plot1}
% \end{figure}
