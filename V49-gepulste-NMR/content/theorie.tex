\section{Zielsetzung}
\label{sec:Zielsetzung}

Ziel des Versuches ist die Bestimmung der Diffusionskonstante von Wasser.
Diese wird mittels gepulster Kernspinresonanz (NMR) ermittelt,
indem also der zeitliche Verlauf einer Magnetisierung der Probe unter
Einstrahlung eines Hochfrequenzfeldes untersucht wird.
Dabei treten Relaxationsprozesse auf, die unter Verwendung zweier
Relaxationszeiten charakterisiert werden.

\section{Theorie}
\label{sec:Theorie}

\subsection{Magnetisierung einer Probe}
\label{sec:TheoMagnetisierung}

Im Folgenden wird die Magnetisierung einer Probe erläutert, die im thermischen
Gleichgewicht mit ihrer Umgebung steht.
Anschließend wird auf die Larmor-Präzession näher eingegangen.

Beim Anlegen eines externen Magnetfelds $\vec{B}_0 = B_0 \vec{e}_\text{B}$ eines ansonsten
feldfreien Raums spalten die Kernspinzustände der Probe in $2S+1$ Unterniveaus auf.
Dabei zeige das Magnetfeld in $z$-Richtung und $S$ bezeichne die Spinquantenzahl der
Zustände, die mittels der Orientierungsquantenzahl $m$ unterschieden werden.
Im thermischen Gleichgewicht sind die Zustände nach der Maxwell-Boltzmann-Verteilung
und damit ungleichmäßig besetzt.
Aufgrund der Orientierung der einzelnen Spins ergibt sich daraus eine Kernspinpolarisation
$\left<S_\text{z}\right>$.
Bei der Betrachtung von Protonen mit $S = \sfrac{1}{2}$ und der Abschätzung
$m \gamma B_0 \ll k_\text{B} T$ ergibt sich in linearer Näherung
\begin{equation*}
  \left<I_\text{z}\right> = -\frac{\hbar^2}{4}\frac{\gamma B_0}{k_\text{B} T}.
\end{equation*}
Dabei bezeichnet $\gamma$ das gyromagnetische Verhältnis des Kerns,
$k_\text{B}$ die Boltzmann-Konstante, $T$ die Temperatur
und $\hbar$ das reduzierte Plancksche Wirkungsquantum.
% TODO: Überlegen, ob hier ein Satz zu m=+-0.5 mit Energieaufspaltung eingefügt werden könnte
Aufgrund der Verknüpfung der Kernspinpolarisationen der Kerne
mit dem magnetischen Moment $\vec{\mu_\text{S}}$ folgt aus der ungleichmäßigen
Besetzung eine makroskopische Magnetisierung $\vec{M_0}$,
deren Erwartungswert in Richtung des äußeren Feldes
\begin{equation*}
  M_0 = \frac{1}{4} \mu_0 \gamma^2 \frac{\hbar^2}{k_\text{B}} N \frac{B_0}{T}
\end{equation*}
beträgt.
Dabei bezeichnet $\mu_0$ die Permeabilität des Vakuums und
$N$ die Anzahl der Momente pro Volumeneinheit.

% Larmor-Präzession
Für die NMR-Spektroskopie ist interessant, wie sich die Magnetisierung $\vec{M}$
der Probe nach einer Auslenkung aus der Gleichgewichtslage $\vec{M_0}$
zeitlich entwickelt.
Aufgrund der großen Anzahl von Einzelmomenten
($N$ in Größenordnung \SI[retain-unity-mantissa=false]{1e28}{\per\cubic\meter})
lässt sich diese Entwicklung klassisch behandeln.
Auf die Magnetisierung der Probe wirkt im externen Magnetfeld ein Drehmoment
$\sim \vec{M} \times \vec{B}$,
welches zu einer Präzessionsbewegung der Magnetisierung um die Achse des
Magnetfelds führt.
Diese Präzession wird Larmor-Präzession und die zugehörige Kreisfrequenz
\begin{equation}
  \omega_\text{L} = \gamma B_0
\end{equation}
wird Larmor-Frequenz genannt.

% Relaxation
Neben der Präzession um das externe Magnetfeld treten Relaxationseffekte auf.
Wird die Magnetisierung aus der Gleichgewichtslage $\vec{M}_0$ entfernt,
strebt sie nach dem Verschwinden der auslösenden Störung wieder zu dieser zurück.
Diese Relaxationseffekte können durch zwei Zeitkonstanten $T_1$ und
$T_2$ charakterisiert werden.
% Beschrieben wird dieser Vorgang mit den Differentialgleichungen
% \begin{align*}
  % \frac{\symup{d} M_\text{z}}{\symup{d} t} &= \frac{M_0 - M_\text{z}}{T_1} \\
  % \frac{\symup{d} M_\text{x}}{\symup{d} t} &= -\frac{M_\text{x}}{T_2} \\
  % \frac{\symup{d} M_\text{y}}{\symup{d} t} &= -\frac{M_\text{y}}{T_2}.
% \end{align*}
Dabei bezeichnet die Zeitkonstante $T_1$ die sogenannte longitudinale oder
Spin-Gitter-Relaxationszeit parallel zur Richtung des externen magnetischen Feldes.
Sie charakterisiert die Zeit, in welcher Energie zwischen dem Kernspinsystem
und Gitterschwingungen ausgetauscht wird. Auch bei flüssigen Proben wird
diese Bezeichnung beibehalten.
Die Größe $T_2$ beschreibt die transversale oder Spin-Spin-Relaxationszeit
senkrecht zur Richtung des externen magnetischen Feldes.
Sie beschreibt die Abnahme der Magnetisierung senkrecht zu $\vec{B}$
durch Wechselwirkungen der Spins mit ihren nächsten Nachbarn.

Werden die Relaxationseffekte und Präzession zusammengefasst, ergeben sich
die sogenannten Blochschen Gleichungen
\begin{equation}
  \begin{split}
    \frac{\symup{d} M_\text{z}}{\symup{d} t} &= \frac{M_0 - M_\text{z}}{T_1} \\
    \frac{\symup{d} M_\text{x}}{\symup{d} t} &= 
      \gamma B_0 M_\text{y} - \frac{M_\text{x}}{T_2} \\
    \frac{\symup{d} M_\text{y}}{\symup{d} t} &=
      \gamma B_0 M_\text{x} - \frac{M_\text{y}}{T_2},
  \end{split}
  \label{eqn:Bloch-Gleichungen}
\end{equation}
welche die zeitliche Entwicklung der Probenmagnetisierung beschreiben.


\subsection{Hochfrequent-Einstrahlungsvorgänge}
\label{sec:HF-Einstrahlung}

Als Auslenkung der Probenmagnetisierung aus ihrer Gleichgewichtslage wird
% ein Hochfrequenzfeld $\vec{B}_\text{HF}$ der Form
% \begin{equation*}
  % \vec{B}_\text{HF} = 2 \vec{B}_1 \cos\!\left(\omega t\right)
  % \quad\quad (\text{mit } \vec{B}_1 \perp \vec{e}_\text{B})
% \end{equation*}
% verwendet.
ein Hochfrequenzfeld $\vec{B}_\text{HF}$ senkrecht zu $\vec{e}_\text{B}$ verwendet.
Es kann als Überlagerung zweier zirkular polarisierten Felder der
Frequenzen $+\omega$ und $-\omega$ aufgefasst werden.
Liegt $+\omega$ in der Nähe der Larmor-Frequenz $\omega_\text{L}$,
lässt sich das Feld zu $-\omega$ vernachlässigen.
Das Hochfrequenzfeld lässt sich somit durch
\begin{equation*}
  B_\text{x} = B_1 \cos\!\left(\omega t\right) \quad\quad
  B_\text{y} = B_1 \sin\!\left(\omega t\right)
\end{equation*}
darstellen.
Zur Lösung der Differentialgleichungen \eqref{eqn:Bloch-Gleichungen} wird in ein
Koordinatensystem transformiert, das mit der Frequenz $\omega$ um $\vec{B}_0$
rotiert. Im neuen System $\left\{x', y', z'\right\}$
ist $\vec{B}_\text{HF}$ zwar konstant (o.B.d.A. in $x'$-Richtung),
jedoch sind die Einheitsvektoren zeitabhängig,
sodass die Differentialgleichung zur Präzession die Gestalt
% \begin{equation*}
  % \frac{\symup{d} \vec{M}}{\symup{d} t} =
  % \gamma \left(\vec{M} \times \vec{B}_\text{ges}\right)
  % -\vec{\omega} \times \vec{M}
% \end{equation*}
% oder
% \begin{equation*}
  % \frac{\symup{d} \vec{M}}{\symup{d} t} =
  % \gamma \left[\vec{M} \times \left(\vec{B}_\text{ges}
  % + \frac{\vec{\omega}}{\gamma}\right)\right]
% \end{equation*}
\begin{equation*}
  \frac{\symup{d} \vec{M}}{\symup{d} t} =
  \gamma \left(\vec{M} \times \vec{V}_\text{eff}\right)
\end{equation*}
mit einem eingeführten effektivem Magnetfeld
\begin{equation*}
  \vec{B}_\text{eff} = \vec{B}_0 + \vec{B}_1 + \frac{\vec{\omega}}{\gamma}
\end{equation*}
hat.
Dies entspricht einer Präzession von $\vec{M}$ um $\vec{B}_\text{eff}$,
was eine Änderung der $z$-Komponente von $\vec{M}$ bedeutet.
Für $\vec{B}_\text{eff} = \vec{B}_1$ präzierdiert die Magnetisierung um $\vec{B}_1$
mit $\sphericalangle\!\left(\vec{M}, \vec{B}_1\right) = \SI{90}{\degree}$.
Wird das Hochfrequenzfeld für die Zeit
\begin{equation}
  \Delta t_{90} = \frac{\pi}{2 \gamma B_1}
  \quad\quad (\text{mit } \Delta t_{90} \ll T_1, T_2)
  \label{eqn:t90}
\end{equation}
eingeschaltet, dreht sich die Magnetisierung aus der $z$-Richtung in die
$y$-Richtung.
Ebenso lässt sich ein \SI{180}{\degree}-Impuls realisieren, der die Magnetisierung
in die negative $z$-Richtung dreht.


\subsection{Der freie Induktionszerfall}
\label{sec:FID}


- Störungen/Verunreinigungen der Probe nicht resonant mit eingestellter Frequenz
  -> gamma anders, also nicht sichtbar, kernselektve Methode

% \begin{figure}
%   \centering
%   \includegraphics[height=5.5cm]{content/Bild.png}
%   \caption{Bilduterschrift}
%   \label{fig:Bild}
% \end{figure}
