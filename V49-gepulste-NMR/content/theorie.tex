\section{Zielsetzung}
\label{sec:Zielsetzung}

Ziel des Versuches ist die Bestimmung der Diffusionskonstante von Wasser.
Diese wird mittels gepulster Kernspinresonanz (NMR) ermittelt,
indem also der zeitliche Verlauf einer Magnetisierung der Probe unter
Einstrahlung eines Hochfrequenzfeldes untersucht wird.
Dabei treten Relaxationsprozesse auf, die unter Verwendung zweier
Relaxationszeiten charakterisiert werden.

\section{Theorie}
\label{sec:Theorie}

\subsection{Magnetisierung einer Probe}
\label{sec:TheoMagnetisierung}

Im Folgenden wird die Magnetisierung einer Probe erläutert, die im thermischen
Gleichgewicht mit ihrer Umgebung steht.
Anschließend wird auf die Larmor-Präzession näher eingegangen.

Beim Anlegen eines externen Magnetfelds $\vec{B} = B_0 \vec{e}_\text{B}$ eines ansonsten
feldfreien Raums spalten die Kernspinzustände der Probe in $2S+1$ Unterniveaus auf.
Dabei zeige das Magnetfeld in $z$-Richtung und $S$ bezeichne die Spinquantenzahl der
Zustände, die mittels der Orientierungsquantenzahl $m$ unterschieden werden.
Im thermischen Gleichgewicht sind die Zustände nach der Maxwell-Boltzmann-Verteilung
und damit ungleichmäßig besetzt.
Aufgrund der Orientierung der einzelnen Spins ergibt sich daraus eine Kernspinpolarisation
$\left<S_\text{z}\right>$.
Bei der Betrachtung von Protonen mit $S = \frac{1}{2}$ und der Abschätzung
$m \gamma B_0 \ll k_\text{B} T$ ergibt sich in linearer Näherung
\begin{equation*}
  \left<I_\text{z}\right> = -\frac{\hbar^2}{4}\frac{\gamma B_0}{k_\text{B} T}.
\end{equation*}
Dabei bezeichnet $\gamma$ das gyromagnetische Verhältnis des Kerns,
$k_\text{B}$ die Boltzmann-Konstante, $T$ die Temperatur
und $\hbar$ das reduzierte Plancksche Wirkungsquantum.
% TODO: Überlegen, ob hier ein Satz zu m=+-0.5 mit Energieaufspaltung eingefügt werden könnte
Aufgrund der Verknüpfung der Kernspinpolarisationen der Kerne
mit dem magnetischen Moment $\vec{\mu_\text{S}}$ wird so eine makroskopische
Magnetisierung $\vec{M_0}$ erzeugt,
deren Erwartungswert in Richtung des äußeren Feldes
\begin{equation*}
  M_0 = \frac{1}{4} \mu_0 \gamma^2 \frac{\hbar^2}{k_\text{B}} N \frac{B_0}{T}
\end{equation*}
beträgt.
Dabei bezeichnet $\mu_0$ die Permeabilität des Vakuums und
$N$ die Anzahl der Momente pro Volumeneinheit.

Für die NMR-Spektroskopie ist interessant, wie sich die Magnetisierung $\vec{M}$
der Probe nach einer Auslenkung aus der Gleichgewichtslage $\vec{M_0}$
zeitlich entwickelt.
Aufgrund der großen Anzahl von Einzelmomenten
($N$ in Größenordnung \SI[retain-unity-mantissa=false]{1e28}{\per\cubic\meter})
lässt sich diese Entwicklung klassisch behandeln.



- Störungen/Verunreinigungen der Probe nicht resonant mit eingestellter Frequenz
  -> gamma anders, also nicht sichtbar, kernselektve Methode

% \begin{figure}
%   \centering
%   \includegraphics[height=5.5cm]{content/Bild.png}
%   \caption{Bilduterschrift}
%   \label{fig:Bild}
% \end{figure}
