\newpage
\section{Durchführung}
\label{sec:Durchführung}

- Frequenz so, dass keine Schwingung sichtbar
  Schwingung dadurch, dass nicht Larmorfrequenz getroffen wurde
  Vorstellbar, dass das rotierende Koordinatensystem falsche Frequenz
- deltaT90 gut einstellen, sodass Halbwertsbreite möglichst groß
  -> möglichst genau die 90° treffen
- Shim ist Einstellung zwischen Gradientenspulen, wird so eingestellt,
  dass Magnetfeld möglichst homogen
  -> zeigt sich, indem FID Kurve möglichst lange sichtbar ist
- P so einstellen, dass mindestens 2 tau. (P Abstand 180° Pulse),
  ansonsten "Antwort des Systems" nicht abgewartet


- Paramagnetische Zentren im Wasser verkürzen T1
  -> Damit schneller messen und nicht 10s warten, bis Probe wieder im GG ist
  - Kalibration, eigentliche Messung mit richtiger Wasserprobe

% \begin{figure}
%   \centering
%   \includegraphics[height=8.0cm]{content/Versuchsaufbau1.png}
%   \caption{Bildunterschrift}
%   \label{fig:Versuchsaufbau1}
% \end{figure}
