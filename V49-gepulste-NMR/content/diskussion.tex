\section{Diskussion}
\label{sec:Diskussion}

- T1 wichtig um zu wissen, wann Probe wieder im thermischen Gleichgewicht.
  Deshalb zuerst gemessen
- Bei richtiger Probe (ohne Paramagnetische Zentren) deltaT90 noch einmal justiert,
  da die param. Probe zu groß ist. Sie reicht damit nicht nur in den homogenen
  Bereich des Magnetfelds, sondern auch noch darüber hinaus.
- Probe sitzt nicht fest, ein Wackeln an der Probe verändert die Position dieser im
  Magnetfeld und das ändert (durch experimentelle Inhomogenität) das Signal DRASTISCH!
