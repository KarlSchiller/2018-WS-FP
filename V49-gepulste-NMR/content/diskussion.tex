\section{Diskussion}
\label{sec:Diskussion}

% TODO: Überarbeiten bzw Korrektur lesen

Zuerst ist zu erwähnen, dass zusätzlich zu der Justage durch eine Probe ohne
paramagnetische Zentren eine Justage nach einlegen der paramagnetischen Probe
erforderlich ist, da diese aus dem homogenen Magnetfeld herausragt und somit die
Messungen beeinflusst. Dieser Einfluss ist bei der Betrachtung allerdings nicht
berücksichtigt worden.

Ebenso sitzt die Probe nicht fest in einer Halterung, sondern ist bewegbar, was
dazu führt, dass die Position der Probe im Magnetfeld geändert wird und somit durch
eine experimentelle Inhomogenität das Signal stört.

Eine weitere Quelle für Unsicherheiten ist, dass häufige Neujustage die Messungen
beinflusst haben und nur annähernd konsistent sind.

Die Messungen von $T_1$ und $T_2$ lieferten folgende Ergebnisse:
\begin{align*}
  T_1 &= \SI{2.037(29)}{\second} \\
  T_2 &= \SI{1.657(17)}{\second}
\end{align*}
Dabei entspricht dieses Ergebniss genau der Erwartung $T_1 > T_2$, da die longitudinale
Relaxations $T_1$ länger andauern sollte als die transversale. Dabei wurde von der
Reihenfolge der Messungen in der Anleitung abgewichen und $T_1$ zuerst gemessen.
Dies bezieht sich darauf, dass die Probe während der Messung wieder in ein thermisches
Gleichgewicht gelangen soll, was durch Erwärmung des Magneten durch einen eintretenden
Stromfluss nach längerem Betrieb erschwert wird.
Aus der Formel der Regression von $T_1$ wurde der Term $+ U_1$ nicht zur Regression
herangezogen, da er die Fitparameter in sofern verändert hat, dass sie eine
Verschechterung darstellten.

Bei Betrachtung der Carr-Purcell-Methode ist eine Bestimmung der transversalen
Relaxationszeit nicht möglich, da bei jeder Pulsgebung eine geringe Phasenverschiebung
auftritt, welche sich zu einer sichtbaren Oszillation in Abbildung \ref{fig:CP2}
addiert.

Die Bestimmung der Viskosität $\eta$ birkt die Unsicherheit, dass die Raumtemperatur
nur schätzungsweise bestimmt wurde. Zudem ist die Zeitmessung durch eine Stopuhr durch
Menschliche Unsicherheiten verfälscht und somit keine exakte Messung. Eine mechanische-/
elektronische Messung würde die Messung deutlich verbessern.

Um die Bestimmung des Molekülradius zu überprüfen wird die Berechnung mit Hilfe der zuvor
bestimmten Diffusionskonstante und der Viskosität jeweils mit den theoretischen
Berechnungen verglichen.
Somit weicht der bestimmte Wert $r$ um \SI{36.7}{\percent} von $r_\text{hcp}$ und um
\SI{7.8}{\percent} von $r_\text{VdW}$ ab.
