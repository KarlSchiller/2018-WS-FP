\section{Diskussion}
\label{sec:Diskussion}

Um in geringerer Zeit die Apparatur auf die Probe justieren zu können,
wurde eine Wasserprobe mit paramagnetischen Zentren verwendet.
Diese hatte jedoch deutlich andere Abmessungen als die eigentliche Wasserprobe
und ragte aus dem homogenen Bereich des Magnetfeldes heraus.
Eine aufgrund dessen auftretende systematische Abweichung der Messwerte ist in der
Auswertung nicht berücksichtigt worden.

Ebenso sitzt die Probe nicht fest in einer Halterung, sondern ist bewegbar, was
dazu führt, dass die Position der Probe im Magnetfeld geändert wird und somit durch
eine experimentelle Inhomogenität das Signal stört.

% Eine weitere Quelle für Unsicherheiten ist, dass häufige Neujustage die Messungen
% beinflusst haben und nur annähernd konsistent sind.

Die Auswertung der Messungen von $T_1$ und $T_2$ lieferten folgende Ergebnisse:
\begin{align*}
  T_1 &= \SI{2.037(29)}{\second} \\
  T_2 &= \SI{1.657(17)}{\second}
\end{align*}
Es zeigt sich, dass $T_1 > T_2$ gilt, was mit den Ergebnissen der Relaxationstheorie
übereinstimmt.

Nach Aussage der Quelle \cite{QAMRI} hängen die einzelnen Relaxationszeiten
unter anderem von der Reinheit des Wassers ab.
Reines Wasser hat der Website zu Folge Relaxationszeiten von
$T_1 = \SI{4}{\second}$ bzw. $T_2 = \SI{2}{\second}$. Dabei hängen $T_1$ und $T_2$
auch von {\sfrac{$\omega_\text{l}$}{B}} und der gegebenen Temperatur ab. Auf der Website
wurde zudem ein \SI{1.5}{\tesla} Magnetfeld verwendet. Der für diese Messung verwendete
Magnet kann allerdings kein so hohes Magnetfeld erzeugen, was zu bestimmten Unterschieden
zwischen den Messungen führt.
Beim Vergleich mit den experimentell bestimmten Werten fällt auf, dass der
Wert für $T_2$ mit einer Abweichung von \SI{17.1}{\percent} relativ
nahe an diesem Wert liegt.
Ebenso zeigt die vergleichsweise hohe Abweichung von $T_1$ um \SI{49.1}{\percent},
dass hier experimentelle Ungenauigkeiten (wie z.B. eine Fehlkalibration des
Versuchsaufbaus) aufgetreten sein müssen.

Aus der Formel der Regression von $T_1$ wurde der Term $+ U_1$ nicht zur Regression
herangezogen, da er die Fitparameter in sofern verändert hat, dass sie eine
Verschechterung darstellten.

Die Bestimmung der Viskosität $\eta$ birkt die Unsicherheit, dass die Raumtemperatur
nur schätzungsweise bestimmt wurde. Zudem ist die Zeitmessung durch eine Stopuhr durch
Menschliche Unsicherheiten verfälscht und somit keine exakte Messung. Eine mechanische-/
elektronische Messung würde die Messung deutlich verbessern.

Um die Bestimmung des Molekülradius zu überprüfen, wird die Berechnung mit Hilfe der zuvor
bestimmten Diffusionskonstante und der Viskosität jeweils mit den
in Abschnitt \ref{sec:AuswMolekuel} berechneten theoretischen Werten verglichen.
Bei Annahme einer hcp-Struktur des Wassers weicht der experimentelle Wert
um \SI{22.4}{\percent} und bei Annahme eines
van-der-Waals Gaases am kritischen Punkt um \SI{7.6}{\percent} ab.
% Somit weicht der bestimmte Wert $r$ um \SI{36.7}{\percent} von $r_\text{hcp}$ und um
% \SI{7.8}{\percent} von $r_\text{VdW}$ ab.
