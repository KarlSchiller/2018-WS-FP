\documentclass[
  bibliography=totoc,     % Literatur im Inhaltsverzeichnis
  captions=tableheading,  % Tabellenüberschriften
  titlepage=firstiscover, % Titelseite ist Deckblatt
  11pt,                   % Fontgröße
]{scrartcl} % Alternative: scrreprt und scrbook

% Paket float verbessern
\usepackage{scrhack}

% Warnung, falls nochmal kompiliert werden muss
\usepackage[aux]{rerunfilecheck}

% unverzichtbare Mathe-Befehle
\usepackage{amsmath}
% viele Mathe-Symbole
\usepackage{amssymb}
% Erweiterungen für amsmath
\usepackage{mathtools}
\usepackage{dsfont}
% Fonteinstellungen
\usepackage{fontspec}
% Latin Modern Fonts werden automatisch geladen
% Alternativ:
%\setromanfont{Libertinus Serif}
%\setsansfont{Libertinus Sans}
%\setmonofont{Libertinus Mono}

% Wenn man andere Schriftarten gesetzt hat, sollte man
% das Seiten-Layout neu berechnen lassen
\recalctypearea

% Rändermaße
\usepackage[
    left=3cm,
    right=3cm,
    top=1.5cm,
    bottom=2cm,
    includeheadfoot
]{geometry}

% deutsche Spracheinstellungen
\usepackage{polyglossia}
\setmainlanguage{german}


\usepackage[
  math-style=ISO,    % ┐
  bold-style=ISO,    % │
  sans-style=italic, % │ ISO-Standard folgen
  nabla=upright,     % │
  partial=upright,   % ┘
  warnings-off={           % ┐
    mathtools-colon,       % │ unnötige Warnungen ausschalten
    mathtools-overbracket, % │
  },                       % ┘
]{unicode-math}

% traditionelle Fonts für Mathematik
\setmathfont{Latin Modern Math}
% Alternativ:
%\setmathfont{Libertinus Math}

\setmathfont{XITS Math}[range={scr, bfscr}]
\setmathfont{XITS Math}[range={cal, bfcal}, StylisticSet=1]

% Zahlen und Einheiten
\usepackage[
  locale=DE,                   % deutsche Einstellungen
  separate-uncertainty=true,   % immer Fehler mit \pm
  per-mode=symbol-or-fraction, % / in inline math, fraction in display math
]{siunitx}

% chemische Formeln
\usepackage[
  version=4,
  math-greek=default, % ┐ mit unicode-math zusammenarbeiten
  text-greek=default, % ┘
]{mhchem}

% richtige Anführungszeichen
\usepackage[autostyle]{csquotes}

% schöne Brüche im Text
\usepackage{xfrac}

% Standardplatzierung für Floats einstellen
\usepackage{float}
\floatplacement{figure}{htbp}
\floatplacement{table}{htbp}

% Floats innerhalb einer Section halten
\usepackage[
  section, % Floats innerhalb der Section halten
  below,   % unterhalb der Section aber auf der selben Seite ist ok
]{placeins}

% Seite drehen für breite Tabellen: landscape Umgebung
\usepackage{pdflscape}

% Captions schöner machen.
\usepackage[
  labelfont=bf,        % Tabelle x: Abbildung y: ist jetzt fett
  font=small,          % Schrift etwas kleiner als Dokument
  width=0.9\textwidth, % maximale Breite einer Caption schmaler
]{caption}
% subfigure, subtable, subref
\usepackage{subcaption}

% Grafiken können eingebunden werden
\usepackage{graphicx}
% In folgenden Ordnern wird nach den Graphiken gesucht
\graphicspath{
  {images/}{../images/}
  {pics/}{../pics}
  {content/}{../content/}
  {build/}{../build/}
  {rohdaten/}{../rohdaten/}
  {python/}{../python}
}

% größere Variation von Dateinamen möglich
\usepackage{grffile}
% verhindert das Bilder in anderen Sections auftrerten
\usepackage{placeins}

% schöne Tabellen
\usepackage{booktabs}

% Verbesserungen am Schriftbild
\usepackage{microtype}

% Literaturverzeichnis
\usepackage[
  backend=biber,
]{biblatex}
% Quellendatenbank
\addbibresource{src.bib}
\addbibresource{programme.bib}

% Hyperlinks im Dokument
\usepackage[
  unicode,        % Unicode in PDF-Attributen erlauben
  pdfusetitle,    % Titel, Autoren und Datum als PDF-Attribute
  pdfcreator={},  % ┐ PDF-Attribute säubern
  pdfproducer={}, % ┘
]{hyperref}
% erweiterte Bookmarks im PDF
\usepackage{bookmark}

% Trennung von Wörtern mit Strichen
\usepackage[shortcuts]{extdash}

\author{
  Patrick Schmidt
  \texorpdfstring{
    \\
    \href{mailto:patrick7.schmidt@udo.edu}{patrick7.schmidt@udo.edu}
  }{}
  \texorpdfstring{\and}{, }
  Karl Schiller
  \texorpdfstring{
    \\
    \href{mailto:karl.schiller@udo.edu}{karl.schiller@udo.edu}
  }{}
}

% keine Lücke am Beginn eines Absatzes lassen
\setlength{\parindent}{0pt}
% Zwischen den Absätzen eine Leerzeile einbauen
\setlength{\parskip}{\baselineskip}

% Füllwort bei SIrange verändern
\sisetup{mode=text,range-phrase={\text{ bis }}}
% Neue Buchstaben bei SIunitx
\sisetup{math-micro=\text{µ},text-micro=µ}
\DeclareSIUnit\adc{ADCC}
\DeclareSIUnit\year{a}
\DeclareSIUnit\day{d}
\DeclareSIUnit\hour{h}

% Bilder von Text umfließen lassen
\usepackage{wrapfig}

% Ermögliche Befehl \mean für einen Mittelwert
\newcommand*\mean[1]{\bar{#1}}

%% EOF %%
