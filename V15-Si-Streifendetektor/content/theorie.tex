\section{Zielsetzung}
\label{sec:Zielsetzung}

Das Ziel dieses Versuches ist es, den Aufbau und die Funktionsweise eines
Silizium-Halbleiterdetektors zu aufzuzeigen und mithilfe verschiedener Messungen dessen
Eigenschaften zu untersuchen. Durch die Betrachtung der Ausleseelektronik und
der Datenverarbeitung soll zudem ein Ausblick auf größere Teilchendetektoren
gegeben werden. Schließlich wird eine $\beta^{-}$-Quelle untersucht.

\section{Theorie}
\label{sec:Theorie}
\subsection{Aufbau eines Inneren Detektors}

Die Si-Streifendetektoren sind bei großen teilchenphysikalischen Detektoren,
wie z.B. dem ATLAS-Detektor am LHC, Teil eines sogennanten \textit{Inneren Detektors (ID)}.
Die Aufgabe des ID ist es die Spuren der ionisierenden Teilchen zu erfassen und
durch die Bestimmung ihres Impulses erste wichtige Informationen über diese zu
erhalten.
Dabei werden dicht an die Pipeline, in der die Ausgangsteilchen beschleunigt und
zur Kollision gebracht werden. Nahe an der Pipeline sind Pixel-Detektoren plaziert,
diese sind sehr ähnlich aufgebaut wie Streifendetektoren. Aufgrund ihrer kleineren
Detektionsfläche besitzen diese eine wesentlich höhere örtliche Auflösung, sind
allerdings in der Produktion erheblich teurer. Daher werden die günstigeren
Streifendetektoren in zueinander verdrehten Schichten über die Pixeldetektoren
gelegt, um die Spuren weiter verfolgen zu können.
Als letzte Schicht wird ein Übergangsstrahlungsspurdetektor verwendet, der die
Aufgabe hat die Krümmung der Teilchenspuren bei einem angelegten Magnetfeld zu
erfassen. Im weiteren Schritt werden die Krümmungsradien der Teilchenspuren
berechnet und daraus die Impulse berechnet.

\subsection{Halbleiter}
Als Halbleiter werden jene Materialien bezeichnet, die werde Isolatoren, noch Leiter
sind. Das bedeutet, dass sie eine Bandlücke mit maximal \SIrange{4}{5}{\electronvolt}
besitzen. Eine Bandlücke ist
als jene Energie definiert, die ein Elektron besitzen muss, um von dem Valenz-
in das Leitungsband zu wechselt. Dabei gibt es verschiedene
Untergruppen der Halbleiter.
In diesem Versuch wird dazu Silizium verwendet. Dieser Halbleiter besitzt vier
Valenzelektronen und eine typische Bandlücke von \SI{1.107}{\electronvolt}.
 Dabei werden die Siliziumatome in einer Diamantstruktur angeordnet. Daraus
 resultiert, dass jedes Atom vier Nachbaratome besitzt, welche alle zur Bindungung
 beitragen.

 Springt ein Elektron in das Leitungsband, so hinterlässt es einen positiv
 geladenen Atomrumpf. Diese positive Restladung wird auch \textit{Loch} genannt,
 welches als Quasiteilchen betrachet wird. Trifft nun ein freies Elektron auf ein
 Loch, so kommt es zu einer Rekombination. Durch eine von außen angelegte Spannung
 kommt es dazu, das eine Rekombination verhindert wird und sich die Elektronen
 zu der Anode, Löcher hingegen zur Kathode bewegen. Somit kommt es zur Eigenleitung
 mit einer geringen Ladungträgerdichte.

 Durch Hinzufügen eines geeigneten Fremdatoms wird die Leitfähigkeit jedoch verbessert.
 Hier wird zwischen p- und n-Typ halbleitern unterschieden.

 Bei p-Typ Halbleitern oder Akzeptoren wird ein Fremdatom mit einer niedrigeren
 Anzahl von Valenzelektronen als der Halbleiter
 hinzugefügt. Dadurch fehlt in der Gitterstruktur zur kovalenten
 Bindung eine bindende negative Ladung und das Loch bewegt sich wie ein freier
 positiver Ladungsträger.

 Die n-Typ Halbleiter oder kennzeichnen sich analog zu dem p-Typ dadurch aus, das ein
 Fremdatom in die Gitterstruktur eingebaut wird. Jedoch besitzt dieses ein
 Valenzelektron mehr und sorgt daher für ein zusätzliches Elektron.

 Ein wichtiger Prozess ist hier ein pn-Übergang. Dieser bezeichnet eine Verbindung
 einer p-dorierten mit einer n-dotierten Schicht innerhalb des Halbleiters. Durch den
 Potentialunterschied der verschieden geladenen schichten entsteht eine sogennante
 Diffusionsspannung $U_\text{D}$. Bei Detektoren wird an die p-Seite ein negatives
 Potential und an der n-Seite ein positives Potential angelegt. Dies sorgt dafür,
 dass sich die ladungsträger zu der Anode/kathode bewegen und am pn-Übergang eine
 ladungträgerarme Zone oder Depletionszone mit einer Depletionsspannung $U_\text{Dep}$
 ausbildet. Die Depletionsspannung liegt dabei typischerweise wesentlich höher als die
 Vorspannung ($U \gg U_\text{Deg}$). Ihre Dicke hängt von der
 angelegten Vorspannung und der effektiven Dotierungskonzentration $N_\text{eff}$.
 \begin{equation}
   N_\text{eff} = \frac{N_\text{D}N_\text{A}}{n_\text{D}+N_\text{A}}
 \end{equation}
Die Parameter $N_\text{D}$ und $N_\text{A}$ bezeichnen die Dotierungskonzetrationen
der p- und n-Schichten.
Daraus ergibt sich die Formel
\begin{equation}
  d(U) = \sqrt{\frac{e\epsilon U}{q N_\text{eff}}}
\end{equation}
mit der Elementarladung $q$ und der Dielektrischen Konstante $\epsilon$ von
Silizium. Das Maximum erreicht die Depletionszone bei $U_\text{Dep}$, daher kann
auch
\begin{equation}
  d(U) \approx \frac{q}{2\epsilon} N_\text{eff}D^2
\end{equation}
geschrieben werden. Dabei ist $D$ die Sensordicke.

Im idealen Fall ist der Detektor vollständig depletiert, d.h. es fließt nur Strom,
wenn ein geladenes Teilchen sich durch den Detektor bewegt. In Realität entstehen
durch thermische Anregeung allerdings Elektron-Loch-Paare, die in der Depletionszone
rekombinieren und so einen Leckstrom verursachen. Bei Silizium liegt die Energie
zur Erzeugung eines Elektronen-Loch-paares bei \SI{3.6}{\kilo\electronvolt}.
In diesem Fall liegt die angelegte Spannung unterhalb $U_\text{Dep}$ und es gilt
dür die Dicke der Depletionszone
\begin{align}
  d_\text{c}(U) &= D\sqrt{\frac{U}{U_\text{Dep}}} \hspace{1.7cm}\text{für } U \less U_\text{Dep} \\
  d_\text{c}(U) &= D \hspace{3cm}\text{für } U \le U_\text{Dep}
\end{align}


\subsection{Wechselwirkungen ionisierender Strahlung}
In diesem Versuch wird eine ${90}^\text{Sr}$-Quelle verwendet. Diese ist eine reiner
$\beta^{-}$-Strahler. Strontium besitzt eine Zerfallskette
\begin{align}
  {90}^\text{Sr} \rightarrow{\beta^{-}} {90}^\text{Y} \rightarrow{\beta^{-}} {90}^\text{Zr},
\end{align}
wobei sowohl Yttrium als auch Zirkonium ebenso nahezu reine $\beta^{-}$-Strahler
sind.
 Dabei zerfallen Neutronen wie folgt in Protonen:
\begin{align}
  \text{n} \rightarrow \text{p} + \text{e}^{-} + \overline{\nu_\text{e}}
\end{align}
Die Zerfallsenergie wird auf alle drei Endprodukte verteilt, wobei die maximale
kinetische Energie der Elektronen bei \SI{0.546}{\mega\electronvolt} liegt.

\subsubsection{Wechselwirkung von Elektronen in Materie}
Bewegt sich ein Elektron in Materie, so gibt es durch elastische Stöße mit Kernen und
Elektronen. Die Energie der durch den $\beta^{-}$-Zerfall induzierten Elektronen
ist in diesem Experiment nicht gro genug um Effekte wie die Bremsstrahlung, die
inelastischen Stöße und der Cherenkov-Strahlung auslösen zu können, daher werden
diese Effekte hier auch nicht betrachtet.
Ein Eindringen von Elektronen in den hier verwendeten Silizium-Detektor wird durch
ihre Energieabgabe innerhalb des Detektors identifiziert. Dabei lässt sich die
durchgeschnittliche Energiedisposition durch eine modifizierte Bethe-Bloch-Gleichung
beschreiben.
\begin{equation}
  -\frac{\text{d}E}{\text{d}x} = 2\pi \text{N}_\text{a} \text{m}_\text{e} \text{c}\rho \frac{\text{Z}}{\text{A}}\frac{1}{\beta^2}\left[\text{ln}\left(\frac{\tau^2(\tau + 2)}{2(I\/\text{m}_\text{e}\text{c}^2)^2}\right) + \text{F}(\tau) - \delta - 2\frac{\text{c}}{\text{Z}}\right]
\end{equation}
In dieser Abwandlung der eigentlichen Bethe-Bloch-Gleichung beschreibt $\tau$ die
kinetische Energie mit der Einheit m$_\text{e}$c$^2$. Der Faktor F($\tau$), welcher
extra für Elektronen eingefügt wird, ist bestimmt durch:
\begin{equation}
  \text{F}(\tau) = 1 - \beta^2 + \frac{\tau^2\/8 - (2\text{r}_e + 1) \ln2}{(\tau + 1)^2} \hspace{2cm}\text{mit } \tau = \gamma - 1
\end{equation}
In reinem Silizium entschricht die Energiedeposition typischerweise
\SI{3.88}{\mega\electronvolt\per\centi\meter}.

\subsubsection{Energiespektrum im Silizium-Detektor}
Bei hinreichender Dicke des Detektors ist das Spektrum der deponierten Energie eines
Elektrons aufgrund des zentraen Grenzwertsatzes durch eine Gaußverteilung gegeben.
Der hier verwendete Detektor erfüllt diese Bedingung bei einer Dicke von
\SI{300}{\micro\meter} jedoch nicht. Daher gibt es nicht genug wechselwirkungen
um den zentralen Grenzwertsatz anwendet zu können. Daher entspricht die Verteilung
eher einer asymmetrischen Landau-Funktion. Dies resultiert auch daraus, dass manche
Sekundarelektronen nicht innerhalb des Detektors gestoppt werden und es daher zu
einem geringen Teil die deponierte Energie nicht korrekt bestimmt werden kann.
Da die Energei der Elektronen des $\beta$-Zerfalls durch eine Verteilung beschreibbar
sind, beschreibt das Spektrum der dopnierten Energie am bsten eine faltung zwischen
einer Gauß- und einer Landau-Funktion. Diese Verteilung ist in Abbildung \ref{fig:faltung}
dagestellt.
\begin{figure}[htb]
  \centering
  \includegraphics[width=0.6\textwidth]{images/Landau.png}
  \caption{Spektrum der Energiedisposition eines Elektron in einem dünnen Detektor. \cite{anleitung}}
  \label{fig:faltung}
\end{figure}

Dabei ist zwischen dem mittleren Energieverlust und dem wahrscheinlichsten Energieverlust
zu unterscheigen. Dies tritt durch die asymmetrische Landau-Funktion auf.
