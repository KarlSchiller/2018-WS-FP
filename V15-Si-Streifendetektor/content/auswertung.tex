\newpage
\section{Auswertung}
\label{sec:Auswertung}

\subsection{Strom-Spannungs-Kennlinie}
\label{sec:UI-Auswerung}

Wie im Theoriekapitel \ref{sec:Halbleiter-Theorie}
beschrieben, ergibt sich aus thermischen Anregungen von Elektron-Loch-Paaren
ein Leckstrom.
Dieser ist für einen Bereich von \SIrange{0}{200}{\volt} aufgenommen worden,
die zugehörigen Messwerte befinden sich in Tabelle \ref{tab:ui-characteristic}.
% Händisch aufgenommene Messwerte der UI-Kennlinie
\input{build/ui-characteristic.tex}
Nach Überschreiten der Depletionsspannung $U_\text{Dep}$ werden keine Ladungsträger mehr als
Leckstrom gemessen, die aus der Dotierung resultieren.
Es tragen nur noch in Abschnitt \ref{sec:Halbleiter-Theorie} beschriebene
thermische Anregungen zum Leckstrom bei.
Somit ergibt sich im Bereich um $U_\text{Dep}$ eine Unstetigkeit der in
Abbildung \ref{fig:ui-characteristic} dargestellten Kennlinie.
\begin{figure}
  \centering
  \includegraphics{build/ui-characteristic.pdf}  % [width=\textwidth]
  \caption{Aufgenommene Strom-Spannungs-Kennlinie.
  In grün ist der abgeschätzte Bereich der Depletionsspannung $U_\text{Dep}$ eingezeichnet.
  Die für folgende Messungen eingestellte Spannung $U_\text{Exp}$ ist gestrichelt dargestellt.}
  \label{fig:ui-characteristic}
\end{figure}
Eine Bestimmung der Depletionsspannung ist nur auf einen Bereich zwischen
\SIrange[range-phrase={\text{ und }}]{65}{85}{\volt}
möglich, der in Abbildung \ref{fig:ui-characteristic} grün eingefärbt ist.
Für weitere Messungen wurde die Spannung auf \SI{100}{\volt} eingestellt, um
die vollständige Depletion der Diode zu erreichen.

\FloatBarrier
\subsection{Störsignale}
\label{sec:Stoersignale-Auswertung}

Vor der Kalibration der Detektorstreifen werden diese auf Störsignale untersucht.
Bei \num{1000} Events ohne externe Signalquelle wurden Pedestal und Common Mode Shift
nach \eqref{eqn:pedestal} und \eqref{eqn:common-mode} berechnet.
Aus diesen lässt sich im Anschluss das Noise nach \eqref{eqn:noise} bestimmen.
In Abbildung \ref{fig:pedestal} sind Pedestal und Noise pro Streifen dargestellt,
wobei auch im Folgenden \si{\adc} die ADC Counts bezeichnet.
Während der Pedestal um \SI{512}{\adc} schwankt, liegt der Noise bei grob
\SI{2}{\adc} mit kleinen Erhöhungen an den Randstreifen.
\begin{figure}
  \centering
  \includegraphics{build/pedestal.pdf}  % [width=\textwidth]
  \caption{Pedestal und Noise der einzelnen Streifen bei einer Messung ohne externe
  Signalquelle.}
  \label{fig:pedestal}
\end{figure}
Der Common Mode Shift ist in Abbildung \ref{fig:common-mode} dargestellt und
um \SI{0}{\adc} gaußverteilt.
\begin{figure}
  \centering
  \includegraphics{build/common-mode.pdf}  % [width=\textwidth]
  \caption{Histogramm des Common Mode Shift der einzelnen Messungen ohne externe
  Signalquelle.}
  \label{fig:common-mode}
\end{figure}

\FloatBarrier
\subsection{Kalibration des Streifendetektors}
\label{sec:Kalibration-Auswertung}

Die aufgenommenen Messwerte des Delay-Scans sind in Abbildung \ref{fig:delay-scan}
dargestellt und ergeben eine optimale Verzögerung zwischen Triggersignal und
Chipauslese von \SI{64}{\nano\second}.
\begin{figure}
  \centering
  \includegraphics{build/delay-scan.pdf}  % [width=\textwidth]
  \caption{Aufgenommene Messwerte des Delay-Scans.}
  \label{fig:delay-scan}
\end{figure}

Zur Kalibration bei einer Spannung von \SI{100}{\volt} wurden die Kanäle
10, 35, 60, 90 und 120 verwendet.
Vier sind in Abbildung \ref{fig:calibration} dargestellt und zeigen nur wenig
Abweichung untereinander (Kanal 60 ist in Abbildung \ref{fig:vergleich} dargestellt).
Die Anzahl injizierter Ladungsimpulse lässt sich durch den in Abschnitt \ref{sec:Halbleiter-Theorie}
angegebenen Faktor \SI{3.6}{\electronvolt} (Energie zur Erzeugung eines
Elektron-Loch-Paares) in eine injizierte Energie umrechnen.
\begin{figure}
  \centering
  \includegraphics{build/calibration.pdf}  % [width=\textwidth]
  \caption{Aufgenommene Kalibrationskurven bei $U = \SI{100}{\volt}$.}
  \label{fig:calibration}
\end{figure}
Diese fehlende Abweichung motiviert die Bildung des Mittelwerts der einzelnen ADC Counts,
welcher dann zur Umrechnung der Counts in Energie für alle Streifen verwendet wird.
Zur Umrechnung wurde ein Polynom der Form
\begin{equation}
  E\!\left(ADCC\right) = a_4 ADCC^4 + a_3 ADCC^3 + a_2 ADCC^2 + a_1 ADCC + a_0
  \quad\text{ mit }\quad\left[E\right] = \si{\electronvolt}
  \label{eqn:Umrechnung-in-Energie}
\end{equation}
verwendet, welches mit der Funktion \texttt{curve\_fit} des Pakets
\texttt{scipy.optimize} an die Mittelwerte der Kalibrationskurven regressiert wurde.
Dabei wurde jedoch der Regressionsbereich auf \SIrange{0}{220}{\adc} eingeschränkt.
In Abbildung \ref{fig:umrechnung} ist die Regression mit den Mittelwerten dargestellt,
die regressierten Parameter $a_\text{i}$ ergaben sich zu
\begin{align*}
  a_4 &= \SI{7.5(5)e-05}{\electronvolt\raiseto{-4}\adc} \\
  a_3 &= \SI{-0.017(2)}{\electronvolt\raiseto{-3}\adc} \\
  a_2 &= \SI{2.2(3)}{\electronvolt\raiseto{-2}\adc} \\
  a_2 &= \SI{1013(17)}{\electronvolt\raiseto{-1}\adc} \\
  a_1 &= \SI{-313(282)}{\electronvolt}
\end{align*}
\begin{figure}
  \centering
  \includegraphics{build/umrechnung.pdf}  % [width=\textwidth]
  \caption{Mittelwert der kalibrierten Streifen mit Regression zur Umrechnung
  der \si{\adc} in \si{\electronvolt}.}
  \label{fig:umrechnung}
\end{figure}

Für den Kanal 60 ist in Abbildung \ref{fig:vergleich} die Abweichung der Kalibrationskurve
für einen nicht vollständig depletierten Sensor dargestellt.
\begin{figure}
  \centering
  \includegraphics{build/vergleich.pdf}  % [width=\textwidth]
  \caption{Vergleich der Kalibrationskurven bei verschiedenen angelegten Spannungen
    an Kanal 60.}
  \label{fig:vergleich}
\end{figure}

\FloatBarrier
\subsection{Vermessung der Streifensensoren}
\label{sec:Vermessung-Auswertung}

Zur Vermessung der Silizium-Streifensensoren wird ein in der Kontrolleinheit erzeugter
Laser verwendet. Zur Synchronisation des Lasers mit dem System wird die
optimale Verzögerung zwischen Lasersignal und Chipauslese ermittelt.
Dazu ist in Abbildung \ref{fig:laser-delay} die Verzögerung gegen die ADC Counts
der Auslese aufgetragen.
Das Maximum wird bei einer Verzögerung von \SI{100}{\nano\second} ermittelt und
dieser Wert wird für die eigentliche Vermessung verwendet.
\begin{figure}
  \centering
  \includegraphics{build/laser-delay.pdf}  % [width=\textwidth]
  \caption{Verzögerung zwischen Lasersignal und Chipauslese gegen ADC Counts zur
  Synchronisation.}
  \label{fig:laser-delay}
\end{figure}

Nach der in Abschnitt \ref{sec:Vermessung_Laser} beschriebenen vertikalen Fokussierung des
Lasers wird dieser in \SI{10}{\micro\meter}-Schritten horizontal bewegt,
angefangen bei \SI{4.00}{\milli\meter}.
Die aufgenommenen ADC Counts pro Messposition sind in Abbildung \ref{fig:streifen-uebersicht}
pro Streifen dargestellt.
Dabei zeigen die Streifen 81 und 82 Maxima, sodass hier der Laser bei der jeweiligen
Messposition auf diese Streifen ausgerichtet sein muss.
\begin{figure}
  \centering
  \includegraphics{build/streifen-uebersicht.pdf}  % [width=\textwidth]
  \caption{ADC Counts der einzelnen Streifen pro Messposition des fokussierten
  Lasers.}
  \label{fig:streifen-uebersicht}
\end{figure}

Aus diesem Grund sind diese beiden Streifen in Abbildung \ref{fig:streifen}
genauer dargestellt. Der \emph{pitch} der Streifen beschreibt deren Dicke
und kann aus den Abständen der Signalmaxima zwischen den beiden analysierten
Streifen bestimmt werden.
% Nach Definition beträgt die Dicke der Streifen nicht deren p-Dotierte Dicke,
% sondern die Hälfte dazwischen auch mit dazu.
% Somit grenzen sie aneinander und mit der Periodizität wird quasi die Breite eines
% solchen Streifens ausgemessen.
Unter erneuter Verwendung von \texttt{curve\_fit} und Mittelung über beide Streifen
beträgt der pitch \SI{160(10)}{\micro\meter}, wobei der Fehler aus der gewählten
Schrittweite zwischen den Messpunkten resultiert.
Schließlich kann aus den steigenden und fallenden Flanken der Signalmaxima die
Breite des Laserpunktes grob abgeschätzt werden.
Da der Laser fokussiert ist zeigen die Flanken an, wie viel Weg es braucht, bis
sich der komplette Laserspot auf einem Streifen befindet.
Hier betragen die Flanken mit Augenmaß 2 Messpunkte, sodass unter Berücksichtigung
der gewählten Schrittweite eine Laserbreite von \SI{20(10)}{\micro\meter} resultiert.
\begin{figure}
  \centering
  \includegraphics{build/streifen.pdf}  % [width=\textwidth]
  \caption{ADC Counts der Streifen 81 und 82 pro Messposition des fokussierten
  Lasers. Die Maxima der Signale sind durch gestrichelte Linien markiert.}
  \label{fig:streifen}
\end{figure}

\FloatBarrier
\subsection{Bestimmung der CCE unter Verwendung eines Lasers}
\label{sec:CCEL-Auswertung}

Wie in Abschnitt \ref{sec:CCE} beschrieben wird für verschiedene angelegte Spannungen
das Signal des vom Laser angestrahlten Kanals gemessen.
Die horizontale Stellschraube des Lasers stand dabei auf \SI{4.00}{\milli\meter},
das entspricht in Abbildung \ref{fig:streifen-uebersicht} Messposition \num{0}
und somit ist der hier angesprochene Streifen Nummer \num{83}.
In Abbildung \ref{fig:ccel} sind die aufgenommenen normierten Messwerte dargestellt.
Zusätzlich ist der in Kapitel \ref{sec:UI-Auswerung} abgeschätzte Bereich für die
Depletionsspannung in grün eingetragen.
Anhand der aufgenommenen Messwerte ist der Beginn des Plateaus mit dem Auge schwer
ab zu schätzen, passt jedoch mit dem grünen Bereich zusammen.
An diese Werte wurde im Bereich von \SIrange{0}{70}{\volt} eine Regression
der Form \eqref{eqn:CCEL} mit Hilfe der Funktion \texttt{curve\_fit} des Pakets
\texttt{scipy.optimize} durchgeführt.
Dabei wurde die Depletionsspannung als Schätzungsparameter auf \SI{70}{\volt}
eingestellt. Ohne diese Einschränkung ist keine adequate Regression möglich,
was sich z.B. an den Standardabweichungen der regressierten Parameter zeigt,
die deutlich höher als die eigentlichen Werte der Parameter sind.
In Abbildung \ref{fig:ccel} ist die Regression mit den für die Regression verwendeten
Messwerten in orange dargestellt und die mittlere Eindringtiefe des Lasers
ergab sich zu
\begin{equation*}
  a = \SI{91(5)}{\micro\meter}.
\end{equation*}
\begin{figure}
  \centering
  \includegraphics{build/ccel.pdf}  % [width=\textwidth]
  \caption{Normierte Messwerte in Abhängigkeit der angelegten Spannung von Kanal
  83. Für die Regression (durchgängige Kurve) verwendete Messwerte sind in orange markiert.
  Der grüne Bereich kennzeichnet die in Abschnitt \ref{sec:UI-Auswerung} abgeschätzte
  Depletionsspannung.}
  \label{fig:ccel}
\end{figure}

\FloatBarrier
\subsection{Bestimmung der CCE unter Verwendung einer \texorpdfstring{\ce{^90Sr}}{Sr}-Quelle}
\label{sec:CCEQ-Auswertung}

Für die Bestimmung der CCE mit Hilfe der \ce{^90Sr}-Quelle werden pro angelegter
Spannung und pro Event die \si{\adc} aller Streifen aufaddiert, deren
Signal zu Untergrund Verhältnis größer als \num{5} ist.
Im Anschluss werden alle dieser Events für eine angelegte Spannung gemittelt und
in Abbildung \ref{fig:cce-vergleich} in schwarz dargestellt.
Hier zeigt sich im Vergleich der in grün dargestellten CCE-Auswertung unter
Verwendung des Lasers, dass sich der Beginn der Plateaus ungefähr im selben Bereich
befindet und die CCE der Quelle im Bereich vor dieser Depletionsspannung kleiner
ist.
\begin{figure}
  \centering
  \includegraphics{build/cce-vergleich.pdf}  % [width=\textwidth]
  \caption{Vergleich der verschiedenen Messungen zur CCE.}
  \label{fig:cce-vergleich}
\end{figure}

Die Elektronen haben eine Energie, bei welcher der Verlauf der deponierten Energie
pro Wegstrecke nach der modifizierten Bethe-Bloch-Gleichung ein Minimum aufweist.
Somit deponieren die Elektronen relativ wenig Energie im Detektor.
Weiterhin ist der Sensor mit \SI{300}{\micro\meter} relativ dünn.
Als grobe Abschätzung kann also angenommen werden, dass die Elektronen auf dem
Weg durch den Detektor überall die selbe Energiemenge deponieren.
Somit hängt die Variation der \si{\adc} ausschließlich an der unvollständigen
Depletion der Streifen. Deshalb ist der Übergang des Plateaus in Abbildung
\ref{fig:cce-vergleich} \enquote{schärfer} als bei der Messung mit der Laserquelle.

Nach Formel \eqref{eqn:dicke-depletionszone} ist die Breite der Depletionszone
propotional zur Wurzel der angelegten Spannung.
Zur Überprüfung dieses Zusammenhangs ist in Abbildung \ref{fig:cceq-abhaengigkeit}
das Quadrat des in Abbildung \ref{fig:cce-vergleich} dargestellten Messsignals
für den Bereich der Vorspannung bis \SI{70}{\volt} aufgetragen.
Eine nicht quantitativ ausgewertete lineare Regression zeigt, dass der Verlauf
linear ist.
\begin{figure}
  \centering
  \includegraphics{build/cceq-abhaengigkeit.pdf}  % [width=\textwidth]
  \caption{Das quadrierte normierte Messsignal der Bestimmung der CCE unter Verwendung
  einer \ce{^90Sr}-Quelle für den Spannungsbereich vor der Depletionsspannung.
  In grün ist eine nicht näher betrachtete lineare Regression eingezeichnet.}
  \label{fig:cceq-abhaengigkeit}
\end{figure}

\FloatBarrier
\subsection{Quellenscan einer \texorpdfstring{\ce{^90Sr}}{Sr}-Quelle}
\label{sec:Quellenscan-Auswertung}

Mit Hilfe eines Auswerteskriptes werden Pedestal, Common Mode und die Noise bestimmt
und anhand eines Signal zu Untergrund Verhältnisses von \num{5} Cluster generiert.
In Abbildung \ref{fig:number-of-clusters} ist die Anzahl an auftretenden Clustern gegen die
Häufigkeit aufgetragen.
Am Häufigsten trat \num{1} Cluster pro Event auf.
\begin{figure}
  \centering
  \includegraphics{build/number-of-clusters.pdf}  % [width=\textwidth]
  \caption{Häufigkeit der Anzahl an Cluster pro Event.}
  \label{fig:number-of-clusters}
\end{figure}
Analog ist auch die Anzahl an Streifen bzw. Kanälen pro Cluster in Abbildung
\ref{fig:number-of-channels} dargestellt.
Meistens waren also ein oder zwei Kanäle an einem Cluster beteiligt.
\begin{figure}
  \centering
  \includegraphics{build/number-of-channels.pdf}  % [width=\textwidth]
  \caption{Häufigkeit der Anzahl an Kanälen pro Cluster.}
  \label{fig:number-of-channels}
\end{figure}
In Abbildung \ref{fig:hitmap} ist die Anzahl an Ereignissen pro Kanal dargestellt,
dies wird Hitmap genannt. Hier zeigt sich die räumliche Ausdehnung der Emission der
Quelle. Im Vergleich dazu scheint der fokussierte Laser nur auf wenige Kanäle,
im besten Fall nur auf einen.
\begin{figure}
  \centering
  \includegraphics{build/hitmap.pdf}  % [width=\textwidth]
  \caption{Anzahl an Ereignissen pro Kanal.}
  \label{fig:hitmap}
\end{figure}

Schließlich ist das Energiespektrum in Einheiten der ADC Counts und in \si{\kilo\electronvolt}
in den Abbildungen \ref{fig:spectrum-adc} und \ref{fig:spectrum-eV}
dargestellt.
Dazu wurden alle angesprochenen Streifen eines Clusters zusammen addiert und dann
die Clusterenergie gegen die relative Häufigkeit aufgetragen.
Wie bereits in den vorherigen Abschnitten wurden alle Einträge mit einem Signal zu
Untergrund Verhältnis kleiner als \num{5} nicht mit in die Auswertung eingezogen.
Hierbei ist zu beachten, dass für das Spektrum in \si{\kilo\electronvolt}-Einheiten
die Einträge der einzelnen Streifen zuerst umgerechnet und dann erst addiert werden,
da die Umrechnungsfunktion \eqref{eqn:Umrechnung-in-Energie} nicht linear ist.
Bei beiden Spektren ist nur ein relevanter Bereich dargestellt, auch bei höheren
Energien bzw. ADC Counts gab es vereinzelt noch eine Detektion.
Für das Energiespektrum in \si{\kilo\electronvolt}-Einheiten ergab sich ein
wahrscheinlichster Wert (MPV) von \SI{84(1)}{\kilo\electronvolt}.
Die Standardabweichung ergibt sich aus der gewählten Binbreite von einem Bin pro
\si{\kilo\electronvolt}.

Die mittlere deponierte Energie (MEL) beträgt \SI{103.72(5)}{\kilo\electronvolt}.
Hier ergibt sich die Unsicherheit aus der Standardabweichung des Mittelwerts.
Wird dieser Wert auf die Dicke der Silizium-Streifensensoren von \SI{300}{\micro\meter}
normiert, ergibt sich eine mittlere Energieabgabe von
\SI{3.45(2)}{\mega\electronvolt\per\centi\meter}.
\begin{figure}
  \centering
  \includegraphics{build/spectrum-adc.pdf}  % [width=\textwidth]
  \caption{Energiespektrum der \ce{^90Sr}-Quelle in \si{\adc} Werten.}
  \label{fig:spectrum-adc}
\end{figure}
\begin{figure}
  \centering
  \includegraphics{build/spectrum-eV.pdf}  % [width=\textwidth]
  \caption{Energiespektrum der \ce{^90Sr}-Quelle in \si{\kilo\electronvolt} Werten.}
  \label{fig:spectrum-eV}
\end{figure}
