\newpage
\section{Auswertung}
\label{sec:Auswertung}

\subsection{Strom-Spannungs-Kennlinie}
\label{sec:UI-Auswerung}

Wie im Theoriekapitel \ref{sec:Halbleiter-Theorie}
beschrieben, ergibt sich aus thermischen Anregungen von Elektron-Loch-Paaren
ein Leckstrom.
Dieser ist für einen Bereich von \SIrange{0}{200}{\volt} aufgenommen worden,
die zugehörigen Messwerte befinden sich in Tabelle \ref{tab:ui-characteristic}.
% Händisch aufgenommene Messwerte der UI-Kennlinie
\input{build/ui-characteristic.tex}
Nach Überschreiten der Depletionsspannung $U_\text{Dep}$ werden keine Ladungsträger mehr als
Leckstrom gemessen, die aus der Dotierung resultieren.
Somit ergibt sich im Bereich um $U_\text{Dep}$ eine Unstetigkeit der in
Abbildung \ref{fig:ui-characteristic} dargestellten Kennlinie.
\begin{figure}
  \centering
  \includegraphics{build/ui-characteristic.pdf}  % [width=\textwidth]
  \caption{Aufgenommene Strom-Spannungs-Kennlinie.
  In grün ist der abgeschätzte Bereich der Depletionsspannung $U_\text{Dep}$ eingezeichnet.
  Die für folgende Messungen eingestellte Spannung $U_\text{Exp}$ ist gestrichelt dargestellt.}
  \label{fig:ui-characteristic}
\end{figure}
Eine Bestimmung der Depletionsspannung ist nur auf einen Bereich zwischen
\SIrange[range-phrase={\text{ und }}]{65}{85}{\volt}
möglich, der in Abbildung \ref{fig:ui-characteristic} grün eingefärbt ist.
Für weitere Messungen wurde die Spannung auf \SI{100}{\volt} eingestellt, um
die vollständige Depletion der Diode zu erreichen.

\FloatBarrier
\subsection{Störsignale}
\label{sec:Stoersignale-Auswertung}

Vor der Kalibration der Detektorstreifen werden diese auf Störsignale untersucht.
Bei \num{1000} Events ohne externe Signalquelle wurden Pedestal und Common Mode Shift
nach \eqref{eqn:pedestal} und \eqref{eqn:common-mode} berechnet.
Aus diesen lässt sich im Anschluss des Noise nach \eqref{eqn:noise} bestimmen.
In Abbildung \ref{fig:pedestal} sind Pedestal und Noise pro Streifen dargestellt,
wobei auch im Folgenden \si{\adc} die ADC Counts bezeichnet.
Während der Pedestal um \SI{512}{\adc} schwankt, liegt der Noise bei grob
\SI{2}{\adc} mit kleinen Erhöhungen an den Randstreifen.
\begin{figure}
  \centering
  \includegraphics{build/pedestal.pdf}  % [width=\textwidth]
  \caption{Pedestal und Noise der einzelnen Streifen bei einer Messung ohne externe
  Signalquelle.}
  \label{fig:pedestal}
\end{figure}
Der Common Mode Shift ist in Abbildung \ref{fig:common-mode} dargestellt und
um \SI{0}{\adc} gaußverteilt.
\begin{figure}
  \centering
  \includegraphics{build/common-mode.pdf}  % [width=\textwidth]
  \caption{Histogramm des Common Mode Shift der einzelnen Messungen ohne externe
  Signalquelle.}
  \label{fig:common-mode}
\end{figure}

\FloatBarrier
\subsection{Kalibration des Streifendetektors}
\label{sec:Kalibration-Auswertung}

Die aufgenommenen Messwerte des Delay-Scans sind in Abbildung \ref{fig:delay-scan}
dargestellt und ergeben eine optimale Verzögerung zwischen Triggersignal und
Chipauslese von \SI{64}{\nano\second}.
\begin{figure}
  \centering
  \includegraphics{build/delay-scan.pdf}  % [width=\textwidth]
  \caption{Aufgenommene Messwerte des Delay-Scans.}
  \label{fig:delay-scan}
\end{figure}

Zur Kalibration bei einer Spannung von \SI{100}{\volt} wurden die Kanäle
10, 35, 60, 90 und 120 verwendet.
Vier sind in Abbildung \ref{fig:calibration} dargestellt und zeigen nur wenig
Abweichung untereinander (Kanal 60 ist in Abbildung \ref{fig:vergleich} dargestellt).
Die Anzahl injizierter Ladungsimpulse lässt sich durch den in Abschnitt \ref{sec:Halbleiter-Theorie}
angegebenen Faktor \SI{3.6}{\electronvolt} (Energie zur Erzeugung eines
Elektron-Loch-Paares) in eine injizierte Energie umrechnen.
\begin{figure}
  \centering
  \includegraphics{build/calibration.pdf}  % [width=\textwidth]
  \caption{Aufgenommene Kalibrationskurven bei $U = \SI{100}{\volt}$.}
  \label{fig:calibration}
\end{figure}
Diese fehlende Abweichung motiviert die Bildung des Mittelwerts der einzelnen ADC Counts,
welcher dann zur Umrechnung der Counts in Energie für alle Streifen verwendet wird.
Zur Umrechnung wurde ein Polynom der Form
\begin{equation*}
  E\!\left(ADCC\right) = a_4 ADCC^4 + a_3 ADCC^3 + a_2 ADCC^2 + a_1 ADCC + a_0
  \quad\text{ mit }\quad\left[E\right] = \si{\electronvolt}
\end{equation*}
verwendet, welches mit der Funktion \texttt{curve\_fit} des Pakets
\texttt{scipy.optimize} an die Mittelwerte der Kalibrationskurven regressiert wurde.
Dabei wurde jedoch der Regressionsbereich auf \SIrange{0}{220}{\adc} eingeschränkt.
In Abbildung \ref{fig:umrechnung} ist die Regression mit den Mittelwerten dargestellt,
die regressierten Parameter $a_\text{i}$ ergaben sich zu
\begin{align*}
  a_4 &= \SI{7.5(5)e-05}{\electronvolt\raiseto{-4}\adc} \\
  a_3 &= \SI{-0.017(2)}{\electronvolt\raiseto{-3}\adc} \\
  a_2 &= \SI{2.2(3)}{\electronvolt\raiseto{-2}\adc} \\
  a_2 &= \SI{1013(17)}{\electronvolt\raiseto{-1}\adc} \\
  a_1 &= \SI{-313(282)}{\electronvolt}
\end{align*}
\begin{figure}
  \centering
  \includegraphics{build/umrechnung.pdf}  % [width=\textwidth]
  \caption{Mittelwert der kalibrierten Streifen mit Regression zur Umrechnung
  der \si{\adc} in \si{\electronvolt}.}
  \label{fig:umrechnung}
\end{figure}

Für den Kanal 60 ist in Abbildung \ref{fig:vergleich} die Abweichung der Kalibrationskurve
für einen nicht vollständig depletierten Sensor dargestellt.
\begin{figure}
  \centering
  \includegraphics{build/vergleich.pdf}  % [width=\textwidth]
  \caption{Vergleich der Kalibrationskurven bei verschiedenen angelegten Spannungen
    an Kanal 60.}
  \label{fig:vergleich}
\end{figure}


\subsection{Vermessung der Streifensensoren}
\label{sec:Vermessung-Auswertung}

- Stellschraube bei \SI{4.00}{\milli\meter}, hier Maximum der Intensität -> Streifen
- 35 Punkte aufgenommen von \SIrange{4.00}{4.34}{\milli\meter}
- pitch der Streifen meint deren Periodizität. Nach Definition beträgt die Dicke der
Streifen nicht deren p-Dotierte Dicke, sondern die Hälfte dazwischen auch mit dazu.
Somit grenzen sie aneinander und mit der Periodizität wird quasi die Breite eines
solchen Streifens ausgemessen.
- Die Ausdehnung des Lasers kann aus der steigenden und fallenden Flanke des Signals
ermittelt werden. Da der Laser fokussiert ist, zeigt es, wie viel Weg es braucht,
bis der komplette Laser auf einen Streifen scheint. Da hier \SI{10}{\micro\meter}-Schritte,
ist dies sehr ungenau abzuschätzen.

\subsection{Bestimmung der Charge Collection Efficiency}
\label{sec:CCE-Auswertung}

- Position des Lasers horizontal wieder bei \SI{4.00}{\milli\meter}, mit vorherigem
Kapitel motivieren.

\subsubsection{Bestimmung der CCE unter Verwendung eines Lasers}
\label{sec:CCEL-Auswertung}

- Die entsprechende Gleichung gilt nur bei Anregung mit Photonen, weshalb dieser Fit
nicht bei der CCEQ-Messung verwendet werden kann
- Depletionsspannung bisher nur sehr ungenau vermessen, deshalb hier bitte als
Fitparameter mitgeben.

\subsubsection{Bestimmung der CCE unter Verwendung einer \texorpdfstring{\ce{^90Sr}}{Sr}-Quelle}
\label{sec:CCEQ-Auswertung}

- Die Elektronen haben eine hohe Energie. Nach der modifizierten Bethe-Bloch-Gleichung
ist bei diesen Energien genau ein Minimum bei der deponierten Energie pro Weg, sodass
relativ wenig Energie deponiert wird.
- Weiterhin dünner Sensor, damit grobe Näherung, dass die Elektronen bei dem Weg durch
den Sensor überall die selbe Energie abgeben.
- Energiedeposition der Elektronen immer die selbe, eine Variation in der gemessenen
deponierten Energie liegt also nur an der angelegten Spannung

\subsubsection{Quellenscan einer \texorpdfstring{\ce{^90Sr}}{Sr}-Quelle}
\label{sec:Quellenscan-Auswertung}

- Signal/Noise = 5 als Cut, alles darunter weg geschnitten. Damit bei Clustern
links und rechts neben dem Hauptpeakt vllt zu viele Informationen heraus genommen.
Daher können Abweichungen entstehen.
- WICHTIG: ACD erst in Ladung/Energiecounts umrechnen, dann Cluster addieren!
Polynom ist nämlich nicht linear...
