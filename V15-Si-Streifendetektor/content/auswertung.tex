\newpage
\section{Auswertung}
\label{sec:Auswertung}

\subsection{Strom-Spannungs-Kennlinie}
\label{sec:UI-Auswerung}

Wie im Theoriekapitel % TODO: Referenz einfügen
beschrieben, ergibt sich aus thermischen Anregungen von Elektron-Loch-Paaren
ein Leckstrom.
Dieser ist für einen Bereich von \SIrange{0}{200}{\volt} aufgenommen worden,
die zugehörigen Messwerte befinden sich in Tabelle \ref{tab:ui-characteristic}.
% Händisch aufgenommene Messwerte der UI-Kennlinie
\input{build/ui-characteristic.tex}
Nach Überschreiten der Depletionsspannung $U_\text{Dep}$ werden keine Ladungsträger mehr als
Leckstrom gemessen, die aus der Dotierung resultieren.
Somit ergibt sich im Bereich um $U_\text{Dep}$ eine Unstetigkeit der in
Abbildung \ref{fig:ui-characteristic} dargestellten Kennlinie.
\begin{figure}
  \centering
  \includegraphics{build/ui-characteristic.pdf}  % [width=\textwidth]
  \caption{Aufgenommene Strom-Spannungs-Kennlinie.
  In grün ist der abgeschätzte Bereich der Depletionsspannung $U_\text{Dep}$ eingezeichnet.
  Die für folgende Messungen eingestellte Spannung $U_\text{Exp}$ ist gestrichelt dargestellt.}
  \label{fig:ui-characteristic}
\end{figure}
Eine Bestimmung der Depletionsspannung ist nur auf einen Bereich zwischen
\SIrange[range-phrase={\text{ und }}]{65}{85}{\volt}
möglich, der in Abbildung \ref{fig:ui-characteristic} grün eingefärbt ist.
Für weitere Messungen wurde die Spannung auf \SI{100}{\volt} eingestellt, um
die vollständige Depletion der Diode zu erreichen.

\FloatBarrier
\subsection{Störsignale}
\label{sec:Stoersignale-Auswertung}

- Common Mode entsteht durch alle möglichen Effekte, die alle Streifen betreffen.
z.B. Kabel, die als Antenne wirken oder das den Versuchsaufbau umgebene Stromnetz
- Common Mode in einem Histogramm darstellen
- Was ist das Nois? Bei dieser Messung gilt die Nullhypothese, dass wir kein Signal haben.
Somit ist das Signal nach Abzug von Pedestal und Common Mode unser Noise

\subsection{Kalibration des Streifendetektors}
\label{sec:Kalibration-Auswertung}

- Maximum des Delay-Scans bei \SI{64}{\nano\second}
- Betrachtete Kanäle 35, 90, 10, 120, 60
- Messwerte der Kanäle und deren Mittelwert plotten. Begründung:
Weil die so wenig voneinander abweichen, nehmen wir diese Kalibration für alle Streifen.
- Definierte Anzahl an Ladungsimpulsen auf X achse. Mit der benötigten Energie
für die erzeugung eines Elektron-Loch-Paares lässt sich dies in eine Energie umrechnen.
- Bei \SI{0}{\volt} Kanal 60

\subsection{Vermessung der Streifensensoren}
\label{sec:Vermessung-Auswertung}

- Stellschraube bei \SI{4.00}{\milli\meter}, hier Maximum der Intensität -> Streifen
- 35 Punkte aufgenommen von \SIrange{4.00}{4.34}{\milli\meter}
- pitch der Streifen meint deren Periodizität. Nach Definition beträgt die Dicke der
Streifen nicht deren p-Dotierte Dicke, sondern die Hälfte dazwischen auch mit dazu.
Somit grenzen sie aneinander und mit der Periodizität wird quasi die Breite eines
solchen Streifens ausgemessen.
- Die Ausdehnung des Lasers kann aus der steigenden und fallenden Flanke des Signals
ermittelt werden. Da der Laser fokussiert ist, zeigt es, wie viel Weg es braucht,
bis der komplette Laser auf einen Streifen scheint. Da hier \SI{10}{\micro\meter}-Schritte,
ist dies sehr ungenau abzuschätzen.

\subsection{Bestimmung der Charge Collection Efficiency}
\label{sec:CCE-Auswertung}

- Position des Lasers horizontal wieder bei \SI{4.00}{\milli\meter}, mit vorherigem
Kapitel motivieren.

\subsubsection{Bestimmung der CCE unter Verwendung eines Lasers}
\label{sec:CCEL-Auswertung}

- Die entsprechende Gleichung gilt nur bei Anregung mit Photonen, weshalb dieser Fit
nicht bei der CCEQ-Messung verwendet werden kann
- Depletionsspannung bisher nur sehr ungenau vermessen, deshalb hier bitte als
Fitparameter mitgeben.

\subsubsection{Bestimmung der CCE unter Verwendung einer \texorpdfstring{\ce{^90Sr}}{Sr}-Quelle}
\label{sec:CCEQ-Auswertung}

- Die Elektronen haben eine hohe Energie. Nach der modifizierten Bethe-Bloch-Gleichung
ist bei diesen Energien genau ein Minimum bei der deponierten Energie pro Weg, sodass
relativ wenig Energie deponiert wird.
- Weiterhin dünner Sensor, damit grobe Näherung, dass die Elektronen bei dem Weg durch
den Sensor überall die selbe Energie abgeben.
- Energiedeposition der Elektronen immer die selbe, eine Variation in der gemessenen
deponierten Energie liegt also nur an der angelegten Spannung

\subsubsection{Quellenscan einer \texorpdfstring{\ce{^90Sr}}{Sr}-Quelle}
\label{sec:Quellenscan-Auswertung}

- Signal/Noise = 5 als Cut, alles darunter weg geschnitten. Damit bei Clustern
links und rechts neben dem Hauptpeakt vllt zu viele Informationen heraus genommen.
Daher können Abweichungen entstehen.
- WICHTIG: ACD erst in Ladung/Energiecounts umrechnen, dann Cluster addieren!
Polynom ist nämlich nicht linear...
