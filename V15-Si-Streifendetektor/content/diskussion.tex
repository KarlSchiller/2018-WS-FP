\section{Diskussion}
\label{sec:Diskussion}

Bei der Analyse der Strom-Spannungs-Kennlinie wurde für die Depletionsspannung ein
Bereich von \SIrange{65}{85}{\volt} abgeschätzt, wobei dies jedoch nach Augenmaß
geschah. In Abbildung \ref{fig:cce-vergleich} sind die beiden CCE-Messungen dargestellt
und suggerieren eine Depletionsspannung am unteren Rand des vorher abgeschätzten Bereiches
bei ungefähr \SI{60}{\volt}.
Dies würde ebenfalls der Herstellerangabe von ungefähr \SI{60}{\volt} entsprechen \cite{alibava}.
Kleinere Abweichungen können zumindestens bei der Lasermessung dadurch entstanden sein,
dass sowohl bei der Einstellung der Diodenspannung, als auch bei der horizontalen
Position des Lasers die jeweils kleinste ablesbare Stelle verwendet wurde und somit
kleinere Abweichungen auftreten können.

Wie erwartet ist der Common Mode Shift um \num{0} verteilt, jedoch zeigt sich bei
der Noise eine leichte Erhöhung an den Randstreifen. Hier werden Rauschsignale
sensitiver detektiert.
Der Vergleich der zwei Kalibrationskurven in Abbildung \ref{fig:vergleich} zeigt,
dass bei einem voll depletierten Sensor mehr ADC Counts bei selber injezierter
Energie ausgegeben werden. Dies deckt sich mit den in der Theorie beschriebenen
Erklärung.
% - Common Mode entsteht durch alle möglichen Effekte, die alle Streifen betreffen.
% z.B. Kabel, die als Antenne wirken oder das den Versuchsaufbau umgebene Stromnetz

Weiterhin ist die Bestimmung der Breite des Laserstrahls mit einer Standardabweichung
von \SI{10}{\micro\meter} relativ ungenau. Dies könnte verbessert werden, indem
mehr Messwerte genommen würden, also der Laser z.B. immer um \SI{1}{\micro\meter}
verschoben wird.
Der in der Versuchsanleitung angegebene pitch der Streifen von \SI{160}{\micro\meter}
\cite[13]{anleitung} konnte bestätigt werden. Aus Abbildung \ref{fig:streifen}
ließ ebenfalls sich ein pitch von \SI{160(10)}{\micro\meter} bestimmen.

Die Depletionsspannung könnte als Regressionsparameter der CCE-Messung unter Verwendung
eines Lasers genauer bestimmt werden. Dazu müssten jedoch weitere Messwerte
unterhalb der Depletionsspannung genommen werden.
Die mittlere Eindringtiefe des Lasers konnte unter Annahme einer Depletionsspannung
von \SI{70}{\volt} auf \SI{91(5)}{\micro\meter} bestimmt werden.

Bei den Messungen unter Verwendung der Quelle ist der Schnitt des Signal zu Untergrund
Verhältnisses bei \num{5} sehr hoch. Damit können aus Clustern links und rechts neben dem
Hauptpeak zu viele Informationen heraus geschnitten werden und Abweichungen der
Clusterenergien nach unten auftreten.
Die mittlere Energieabgabe ergibt sich zu \SI{3.45(2)}{\mega\electronvolt\per\centi\meter}.
Nach Quelle \cite[9]{anleitung} wird eine mittlere Energieabgabe von theoretisch
\SI{3.88}{\mega\electronvolt\per\centi\meter} erwartet.
Damit ergibt sich eine Abweichung von \SI{10.9}{\percent}, die mit dem hohen Schnitt des
Signal zu Untergrund Verhältnisses und der Nichtbeachtung des Zerfalls von \ce{^90Y}
erklärt werden kann.
